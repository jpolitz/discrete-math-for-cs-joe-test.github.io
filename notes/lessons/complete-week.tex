\input{../../resources/lesson-head.tex}

\section*{Before we start}
If you or someone you know is suffering from food and/or housing insecurities 
there are UCSD resources here to help:

Basic Needs Office: \href{https://basicneeds.ucsd.edu/}{https://basicneeds.ucsd.edu/}

Triton Food Pantry (in the old Student Center)
is free and anonymous, and includes produce: 

\href{https://www.facebook.com/tritonfoodpantry/}{https://www.facebook.com/tritonfoodpantry/}

Mutual Aid UCSD: \href{https://mutualaiducsd.wordpress.com/}{https://mutualaiducsd.wordpress.com/}

If you find yourself in an uncomfortable situation, ask for help. 
We are committed to upholding University policies regarding nondiscrimination, sexual violence and sexual harassment.

Counseling and Psychological Services (CAPS) at 858 5343755 or \href{http://caps.ucsd.edu}{http://caps.ucsd.edu}


OPHD at (858) 534-8298, ophd@ucsd.edu , \href{http://ophd.ucsd.edu}{http://ophd.ucsd.edu}. 
CARE at Sexual Assault Resource Center at 858 5345793 sarc@ucsd.edu \href{http://care.ucsd.edu}{http://care.ucsd.edu}

\subsection*{Pandemic resilient instruction}
Fall 2021 is a transition quarter so please be patient with us as we do our best 
to serve the needs of all students while adhering to the university guidelines. 
First and foremost is the health and safety of everyone.  
Please do not come to class if you are sick or even think you might be sick.
Please reach out (minnes@eng.ucsd.edu) if you need support with extenuating circumstances.

Masks are required in class. All students who attend class must also be fully vaccinated against COVID-19
unless they have a university-approved exemption.
Campus policy requires masks and daily ``symptom screeners" for everyone and we expect all students 
to follow these rules. 


\newpage
Welcome to CSE 20: Discrete Math for Computer Science in Fall 2021! 

\section*{Themes and applications for CSE 20}
\begin{itemize}
\item {\bf Technical skepticism}: Know, select and apply appropriate computing knowledge and problem-solving techniques. 
Reason about computation and systems. 
Use mathematical techniques to solve problems. 
Determine appropriate conceptual tools to apply to new situations. 
Know when tools do not apply and try different approaches. 
Critically analyze and evaluate candidate solutions.
\item {\bf Multiple representations}: Understand, guide, shape impact of computing on society/the world. 
Connect the role of Theory CS classes to other applications (in undergraduate CS curriculum and beyond). 
Model problems using appropriate mathematical concepts.
Clearly and unambiguously communicate computational ideas using appropriate formalism. 
Translate across levels of abstraction.
\end{itemize}

{\bf Applications}: Numbers (how to represent them and use them in Computer Science), 
Recommendation systems and their roots in machine learning (with applications like Netflix),
``Under the hood" of computers (circuits, pixel color representation, data structures),
Codes and information (secret message sharing and error correction),
Bioinformatics algorithms and genomics (DNA and RNA).

\section*{Introductions}
Class website: \href{http://cseweb.ucsd.edu/classes/fa21/cse20-a}{http://cseweb.ucsd.edu/classes/fa21/cse20-a}

{\bf Pro-tip}: the URL structure is your map to finding your course website for other CSE classes.

{\bf Pro-tip}: you can use MATH109 to replace CSE20 for prerequisites and other requirements.

Instructor: Prof. Mia Minnes {\tiny{"Minnes" rhymes with Guinness}}, minnes@eng.ucsd.edu, 
\href{http://cseweb.ucsd.edu/~minnes}{http://cseweb.ucsd.edu/~minnes}

Our team: Four TAs and 10 tutors + all of you

Fill in contact info for students around you, if you'd like:
\vspace{50pt}


On a typical week: {\bf MWF} Lectures + review quizzes, {\bf T} HW due, {\bf W} Discussion, office hours, Piazza. 
Project parts will be due some weeks.

All dates are on \href{https://canvas.ucsd.edu/}{Canvas (click for link)} and details are on
 \href{https://discrete-math-for-cs.github.io/website/overview_calendar.html}{course calendar (click for link)}.

Education research: CSE 20 is participating in a project on retention and sense of community 
in UCSD majors; see \href{https://discrete-math-for-cs.github.io/files/CSInclusiveMentoringConsentFormNonCSEDataAnalysis.pdf}{research plan}. If you consent to participate in this study, no action is needed. 
If you DO NOT consent to participate in this study, or you choose to opt-out at any time during the a
cademic year, sign and submit this form to the research contact at retentionstudy@cs.ucsd.edu.


\newpage
\section*{Friday September 24}
\input{../activity-snippets/netflix-intro.tex}
\input{../activity-snippets/ratings-encoding.tex}

{\bf Conclusion}: Modeling involves choosing data types to represent and organize data

\newpage
\subsection*{Review: Week 0 Friday}
\begin{enumerate}
\item Please complete the beginning of the quarter survey \href{https://forms.gle/gvibFnNixxqcWbaU8}{https://forms.gle/gvibFnNixxqcWbaU8}
\item We want you to be familiar with class policies and procedures so you are ready to have a successful quarter. 
Please take a look at the class website http://cseweb.ucsd.edu/classes/fa21/cse20-a
and answer the questions about it on \href{http://gradescope.com}{Gradescope}.
\item Modeling: 
\begin{enumerate}
    \item {\input{../activity-snippets/quiz-ratings-tuples.tex}}
    \item {\input{../activity-snippets/quiz-ratings-count.tex}}
\end{enumerate}
\end{enumerate}
\newpage
\section*{Monday September 27}
\subsection*{Notation and prerequisites}
\documentclass[12pt, oneside]{article}

\usepackage[letterpaper, scale=0.89, centering]{geometry}
\usepackage{fancyhdr}
\setlength{\parindent}{0em}
\setlength{\parskip}{1em}

\pagestyle{fancy}
\fancyhf{}
\renewcommand{\headrulewidth}{0pt}
\rfoot{\href{https://creativecommons.org/licenses/by-nc-sa/2.0/}{CC BY-NC-SA 2.0} Version \today~(\thepage)}

\usepackage{amssymb,amsmath,pifont,amsfonts,comment,enumerate,enumitem}
\usepackage{currfile,xstring,hyperref,tabularx,graphicx,wasysym}
\usepackage[labelformat=empty]{caption}
\usepackage[dvipsnames,table]{xcolor}
\usepackage{multicol,multirow,array,listings,tabularx,lastpage,textcomp,booktabs}

% NOTE(joe): This environment is credit @pnpo (https://tex.stackexchange.com/a/218450)
\lstnewenvironment{algorithm}[1][] %defines the algorithm listing environment
{   
    \lstset{ %this is the stype
        mathescape=true,
        frame=tB,
        numbers=left, 
        numberstyle=\tiny,
        basicstyle=\rmfamily\scriptsize, 
        keywordstyle=\color{black}\bfseries,
        keywords={,procedure, div, for, to, input, output, return, datatype, function, in, if, else, foreach, while, begin, end, }
        numbers=left,
        xleftmargin=.04\textwidth,
        #1
    }
}
{}
\lstnewenvironment{java}[1][]
{   
    \lstset{
        language=java,
        mathescape=true,
        frame=tB,
        numbers=left, 
        numberstyle=\tiny,
        basicstyle=\ttfamily\scriptsize, 
        keywordstyle=\color{black}\bfseries,
        keywords={, int, double, for, return, if, else, while, }
        numbers=left,
        xleftmargin=.04\textwidth,
        #1
    }
}
{}

\newcommand\abs[1]{\lvert~#1~\rvert}
\newcommand{\st}{\mid}

\newcommand{\A}[0]{\texttt{A}}
\newcommand{\C}[0]{\texttt{C}}
\newcommand{\G}[0]{\texttt{G}}
\newcommand{\U}[0]{\texttt{U}}

\newcommand{\cmark}{\ding{51}}
\newcommand{\xmark}{\ding{55}}


\begin{document}
\begin{flushright}
    \StrBefore{\currfilename}{.}
\end{flushright}\begin{center}
\begin{tabular}{|llp{9.8cm}|}
\hline
{\bf Term} & {\bf Notation Example(s)} & {\bf We say in English \ldots } \\
\hline
%$n$-tuple & $(x_1, x_2, x_3)$ & The 3-tuple of $x_1$, $x_2$, and $x_3$ \\
%          & $(3, 4)$ & The 2-tuple or {\bf ordered pair} of $3$ and $4$ \\
sequence & $x_1, \ldots, x_n$ & A sequence $x_1$ to $x_n$ \\
%         & $x_1, \ldots, x_n$ where $n = 0$ & An empty sequence \\
%         & $x_1, \ldots, x_n$ where $n = 1$ & A sequence containing just $x_1$ \\
%         & $x_1, \ldots, x_n$ where $n = 2$ & A sequence containing just $x_1$ and $x_2$ in order \\
%         & $x_1, x_2$ & A sequence containing just $x_1$ and $x_2$ in order \\
summation & $\sum_{i=1}^n x_i$ or $\displaystyle{\sum_{i=1}^n x_i}$ & The sum of the terms of the sequence $x_1$ to $x_n$ \\
&&\\
%maximum & $\displaystyle \max(x, y)$ & The max of $x$ and $y$, when they are numbers \\ % Note that this is different than summation!
%        & $\displaystyle \max_{1 \leq i \leq n} x_i$ & The max of $x_1$ to $x_n$, when they are numbers \\ % Also different from display
%&&\\
%set & & Unordered collection of objects. The set of \ldots \\
all reals & $\mathbb{R}$ & The (set of all) real numbers (numbers on the number line)\\
all integers & $\mathbb{Z}$ & The (set of all) integers (whole numbers including negatives, zero, and positives) \\
all positive integers & $\mathbb{Z}^+$ & The (set of all) strictly positive integers \\
all natural numbers & $\mathbb{N}$ & The (set of all) natural numbers. {\bf Note}: we use the convention that $0$ is a natural number. \\
%roster method & $\{43, 7, 9\}$ & The set whose elements are $43$, $7$, and $9$\\
%              & $\{9, \mathbb{N}\}$ & The set whose elements are $9$ and $\mathbb{N}$\\
%&&\\
%set builder notation & $\{ x \in \mathbb{Z} \mid x > 0\}$ & The set of all $x$ from the integers such that $x$ is greater than $0$ \\
%                     & $\{ 3x  \mid x \in \mathbb{Z} \}$ & The set of all integer multiples of $3$. {\bf Note}: we use the convention that writing two numbers next to each other means multiplication. \\
&&\\
%function rule definition & $f(x) = x + 4$ & Define $f$ of $x$ to be $x + 4$ \\
piecewise rule definition & $f(x) = \begin{cases} x & \text{if~}x \geq 0 \\ -x & \text{if~}x<0\end{cases}$ &
Define $f$ of $x$ to be $x$ when $x$ is nonnegative and to be $-x$ when $x$ is negative\\
function application & $f(7)$ & $f$ of $7$ {\bf or} $f$ applied to $7$ {\bf or} the image of $7$ under $f$\\
                     & $f(z)$ & $f$ of $z$ {\bf or} $f$ applied to $z$ {\bf or} the image of $z$ under $f$\\
                     & $f(g(z))$ & $f$ of $g$ of $z$ {\bf or} $f$ applied to the result of $g$ applied to $z$ \\
&&\\
absolute value & $\lvert -3 \rvert$ & The absolute value of $-3$ \\
square root & $\sqrt{9}$ & The non-negative square root of $9$ \\
&&\\
%summation notation & $\displaystyle \sum_{i=1}^n i$ & The sum of the integers from $1$ to $n$, inclusive \\
%                    & $\displaystyle \sum_{i=1}^n i^2 - 1$ & The sum of $i^2 - 1$ ($i$ squared minus $1$) for each $i$ from $1$ to $n$, inclusive \\
%&&\\
%quotient, integer division & $n~\textbf{div}~m$ & The (integer) quotient upon dividing $n$ by $m$; informally: divide and then 
%drop the fractional part\\
%modulo, remainder & $n~\textbf{mod}~m$ & The remainder upon dividing $n$ by $m$ \\
\hline
\end{tabular}
\end{center}
\end{document}
\subsection*{Data Types: sets, $n$-tuples, and strings}
\documentclass[12pt, oneside]{article}

\usepackage[letterpaper, scale=0.89, centering]{geometry}
\usepackage{fancyhdr}
\setlength{\parindent}{0em}
\setlength{\parskip}{1em}

\pagestyle{fancy}
\fancyhf{}
\renewcommand{\headrulewidth}{0pt}
\rfoot{\href{https://creativecommons.org/licenses/by-nc-sa/2.0/}{CC BY-NC-SA 2.0} Version \today~(\thepage)}

\usepackage{amssymb,amsmath,pifont,amsfonts,comment,enumerate,enumitem}
\usepackage{currfile,xstring,hyperref,tabularx,graphicx,wasysym}
\usepackage[labelformat=empty]{caption}
\usepackage[dvipsnames,table]{xcolor}
\usepackage{multicol,multirow,array,listings,tabularx,lastpage,textcomp,booktabs}

\lstnewenvironment{algorithm}[1][] {   
    \lstset{ mathescape=true,
        frame=tB,
        numbers=left, 
        numberstyle=\tiny,
        basicstyle=\rmfamily\scriptsize, 
        keywordstyle=\color{black}\bfseries,
        keywords={,procedure, div, for, to, input, output, return, datatype, function, in, if, else, foreach, while, begin, end, }
        numbers=left,
        xleftmargin=.04\textwidth,
        #1
    }
}
{}
\lstnewenvironment{java}[1][]
{   
    \lstset{
        language=java,
        mathescape=true,
        frame=tB,
        numbers=left, 
        numberstyle=\tiny,
        basicstyle=\ttfamily\scriptsize, 
        keywordstyle=\color{black}\bfseries,
        keywords={, int, double, for, return, if, else, while, }
        numbers=left,
        xleftmargin=.04\textwidth,
        #1
    }
}
{}

\newcommand\abs[1]{\lvert~#1~\rvert}
\newcommand{\st}{\mid}

\newcommand{\A}[0]{\texttt{A}}
\newcommand{\C}[0]{\texttt{C}}
\newcommand{\G}[0]{\texttt{G}}
\newcommand{\U}[0]{\texttt{U}}

\newcommand{\cmark}{\ding{51}}
\newcommand{\xmark}{\ding{55}}

 
\begin{document}
\begin{flushright}
    \StrBefore{\currfilename}{.}
\end{flushright} \section*{Ratings encoding}


In the table  below,  each row represents a user's ratings of movies: 
\cmark~(check) indicates the person liked the movie, \xmark~(x)
that they didn't, and $\bullet$ (dot) that they didn't rate it one way or 
another (neutral rating or didn't watch). Can encode
these ratings numerically with $1$ for \cmark~(check), $-1$ for \xmark~(x), 
and $0$ for $\bullet$ (dot).

\begin{center}
\begin{tabular}{c|ccc||c}
Person & Fyre & Frozen II & Picard & Ratings written as a  $3$-tuple\\
\hline
$P_1$     & \xmark & $\bullet$ & \cmark & \phantom{$(-1, 0, 1)$} \\
$P_2$     & \cmark & \cmark & \xmark & \phantom{$(1, 1, -1)$} \\
$P_3$     & \cmark & \cmark & \cmark & \phantom{$(1, 1, 1)$} \\
$P_4$     & $\bullet$ & \xmark & \cmark &  \\
\end{tabular}
\end{center} \vfill
\section*{Definitions}


\begin{center}
\begin{tabular}{|llp{9.8cm}|}
\hline
{\bf Term} & {\bf Notation Example(s)} & {\bf We say in English \ldots } \\
\hline
sequence & $x_1, \ldots, x_n$ & A sequence $x_1$ to $x_n$ \\
summation & $\sum_{i=1}^n x_i$ or $\displaystyle{\sum_{i=1}^n x_i}$ & The sum of the terms of the sequence $x_1$ to $x_n$ \\
&&\\
all reals & $\mathbb{R}$ & The (set of all) real numbers (numbers on the number line)\\
all integers & $\mathbb{Z}$ & The (set of all) integers (whole numbers including negatives, zero, and positives) \\
all positive integers & $\mathbb{Z}^+$ & The (set of all) strictly positive integers \\
all natural numbers & $\mathbb{N}$ & The (set of all) natural numbers. {\bf Note}: we use the convention that $0$ is a natural number. \\
&&\\
piecewise rule definition & $f(x) = \begin{cases} x & \text{if~}x \geq 0 \\ -x & \text{if~}x<0\end{cases}$ &
Define $f$ of $x$ to be $x$ when $x$ is nonnegative and to be $-x$ when $x$ is negative\\
function application & $f(7)$ & $f$ of $7$ {\bf or} $f$ applied to $7$ {\bf or} the image of $7$ under $f$\\
                     & $f(z)$ & $f$ of $z$ {\bf or} $f$ applied to $z$ {\bf or} the image of $z$ under $f$\\
                     & $f(g(z))$ & $f$ of $g$ of $z$ {\bf or} $f$ applied to the result of $g$ applied to $z$ \\
&&\\
absolute value & $\lvert -3 \rvert$ & The absolute value of $-3$ \\
square root & $\sqrt{9}$ & The non-negative square root of $9$ \\
&&\\


\hline
\end{tabular}
\end{center} \vfill
\section*{Data types}


\begin{center}
    \begin{tabular}{p{4.6in}p{2.6in}}
    {\bf  Term} & {\bf Examples}:\\
    &  (add additional examples from class)\\
    \hline 
    {\bf set} \newline
    unordered collection of elements & $7 \in \{43, 7, 9 \}$ \qquad $2 \notin \{43, 7, 9 \}$ \\
    {\it repetition doesn't matter} & \\
    {\it Equal sets agree on membership of all elements}& \\
    \hline
    {\bf $n$-tuple} \newline
    ordered sequence of elements with $n$ ``slots" ($n >0$) & \\
    {\it repetition matters, fixed length} &\\
    {\it Equal $n$-tuples have corresponding components equal}& \\
    \hline
    {\bf string} \newline
    ordered finite sequence of elements each from specified
    set & \\
    {\it repetition matters, arbitrary finite length} &\\
    {\it Equal strings have same length and corresponding characters equal}
    \end{tabular}
\end{center}

{\it Special cases}: 

When $n=2$, the 2-tuple is called an {\bf ordered pair}.

A string of length $0$ is called the {\bf empty string} and is denoted $\lambda$.

A set with no elements is called the {\bf empty set} and is denoted $\{\}$ or $\emptyset$. \vfill
\section*{Defining sets}


{\it To define sets:}

To define a set using {\bf roster method}, explicitly list its elements. That is,
start with $\{$ then list elements of 
the set separated by commas and close with $\}$.

To define a set using {\bf set builder definition}, either form 
``The set of all $x$ from the universe $U$ such that $x$ is ..." by writing
\[\{x \in U \mid ...x... \}\]
or form ``the collection of all outputs of some operation when the input ranges over the universe $U$"
by writing
\[\{ ...x... \mid x\in U \}\]

We use the symbol $\in$ as ``is an element of'' to indicate membership in a set.\\


{\bf Example sets}: For each of the following, identify whether it's defined using the roster method
or set builder notation and give an example element.
\begin{itemize}
    \item[]$\{ -1, 1\}$\\
    \item[]$\{0, 0 \}$\\
    \item[]$\{-1, 0, 1 \}$\\
    \item[]$\{(x,x,x) \mid x \in \{-1,0,1\} \}$\\
    \item[]$\{ \}$\\
    \item[]$\{ x \in \mathbb{Z} \mid x \geq 0 \}$\\
    \item[]$\{ x \in \mathbb{Z}  \mid x > 0 \}$\\
    \item[]$\{\A,\C,\U,\G\}$ \\
    \item[]$\{\A\U\G, \U\A\G, \U\G\A, \U\A\A \}$\\
\end{itemize}
 \vfill
\section*{Defining functions ratings}


Recall our representation of Netflix users' ratings of movies as $n$-tuples, where
$n$ is the number of movies in the database. 
Each component of the $n$-tuple is $-1$ (didn't like the movie), $0$ 
(neutral rating or didn't watch the movie), or $1$ (liked the movie).

Consider the ratings $P_1 = (-1, 0, 1)$, $P_2 = (1, 1, -1)$, $P_3 = (1, 1, 1)$,
$P_4 = (0,-1,1)$


Which of $P_1$, $P_2$, $P_3$ has movie preferences most similar to $P_4$?

One approach to answer this question: use {\bf functions} to define distance between user preferences.

For example, consider the function 
$d_0: \phantom{the Cartesian product of the set of ratings on 3 movies with itself} \to \phantom{\mathbb{R}}$
given by
\[
d_0 (~(~ (x_1, x_2, x_3), (y_1, y_2, y_3) ~) ~) = \sqrt{ (x_1 - y_1)^2 + (x_2 - y_2)^2 + (x_3 -y_3)^2}
\]


\vfill
\vfill


{\it Extra example:} A new movie is released, and $P_1$ and $P_2$ watch it before $P_3$, and give it
ratings; $P_1$ gives \cmark~and $P_2$ gives \xmark.
Should this movie be recommended to $P_3$? Why or why not?

{\it Extra example:} Define a new function that could be used to compare the $4$-tuples of ratings encoding
movie preferences now that there are four movies in the database.

\vfill
\newpage \vfill
\section*{Ratings encoding}


In the table  below,  each row represents a user's ratings of movies: 
\cmark~(check) indicates the person liked the movie, \xmark~(x)
that they didn't, and $\bullet$ (dot) that they didn't rate it one way or 
another (neutral rating or didn't watch). Can encode
these ratings numerically with $1$ for \cmark~(check), $-1$ for \xmark~(x), 
and $0$ for $\bullet$ (dot).

\begin{center}
\begin{tabular}{c|ccc||c}
Person & Fyre & Frozen II & Picard & Ratings written as a  $3$-tuple\\
\hline
$P_1$     & \xmark & $\bullet$ & \cmark & \phantom{$(-1, 0, 1)$} \\
$P_2$     & \cmark & \cmark & \xmark & \phantom{$(1, 1, -1)$} \\
$P_3$     & \cmark & \cmark & \cmark & \phantom{$(1, 1, 1)$} \\
$P_4$     & $\bullet$ & \xmark & \cmark &  \\
\end{tabular}
\end{center} \vfill
\section*{Definitions}


\begin{center}
\begin{tabular}{|llp{9.8cm}|}
\hline
{\bf Term} & {\bf Notation Example(s)} & {\bf We say in English \ldots } \\
\hline
sequence & $x_1, \ldots, x_n$ & A sequence $x_1$ to $x_n$ \\
summation & $\sum_{i=1}^n x_i$ or $\displaystyle{\sum_{i=1}^n x_i}$ & The sum of the terms of the sequence $x_1$ to $x_n$ \\
&&\\
all reals & $\mathbb{R}$ & The (set of all) real numbers (numbers on the number line)\\
all integers & $\mathbb{Z}$ & The (set of all) integers (whole numbers including negatives, zero, and positives) \\
all positive integers & $\mathbb{Z}^+$ & The (set of all) strictly positive integers \\
all natural numbers & $\mathbb{N}$ & The (set of all) natural numbers. {\bf Note}: we use the convention that $0$ is a natural number. \\
&&\\
piecewise rule definition & $f(x) = \begin{cases} x & \text{if~}x \geq 0 \\ -x & \text{if~}x<0\end{cases}$ &
Define $f$ of $x$ to be $x$ when $x$ is nonnegative and to be $-x$ when $x$ is negative\\
function application & $f(7)$ & $f$ of $7$ {\bf or} $f$ applied to $7$ {\bf or} the image of $7$ under $f$\\
                     & $f(z)$ & $f$ of $z$ {\bf or} $f$ applied to $z$ {\bf or} the image of $z$ under $f$\\
                     & $f(g(z))$ & $f$ of $g$ of $z$ {\bf or} $f$ applied to the result of $g$ applied to $z$ \\
&&\\
absolute value & $\lvert -3 \rvert$ & The absolute value of $-3$ \\
square root & $\sqrt{9}$ & The non-negative square root of $9$ \\
&&\\


\hline
\end{tabular}
\end{center} \vfill
\section*{Data types}


\begin{center}
    \begin{tabular}{p{4.6in}p{2.6in}}
    {\bf  Term} & {\bf Examples}:\\
    &  (add additional examples from class)\\
    \hline 
    {\bf set} \newline
    unordered collection of elements & $7 \in \{43, 7, 9 \}$ \qquad $2 \notin \{43, 7, 9 \}$ \\
    {\it repetition doesn't matter} & \\
    {\it Equal sets agree on membership of all elements}& \\
    \hline
    {\bf $n$-tuple} \newline
    ordered sequence of elements with $n$ ``slots" ($n >0$) & \\
    {\it repetition matters, fixed length} &\\
    {\it Equal $n$-tuples have corresponding components equal}& \\
    \hline
    {\bf string} \newline
    ordered finite sequence of elements each from specified
    set & \\
    {\it repetition matters, arbitrary finite length} &\\
    {\it Equal strings have same length and corresponding characters equal}
    \end{tabular}
\end{center}

{\it Special cases}: 

When $n=2$, the 2-tuple is called an {\bf ordered pair}.

A string of length $0$ is called the {\bf empty string} and is denoted $\lambda$.

A set with no elements is called the {\bf empty set} and is denoted $\{\}$ or $\emptyset$. \vfill
\section*{Defining sets}


{\it To define sets:}

To define a set using {\bf roster method}, explicitly list its elements. That is,
start with $\{$ then list elements of 
the set separated by commas and close with $\}$.

To define a set using {\bf set builder definition}, either form 
``The set of all $x$ from the universe $U$ such that $x$ is ..." by writing
\[\{x \in U \mid ...x... \}\]
or form ``the collection of all outputs of some operation when the input ranges over the universe $U$"
by writing
\[\{ ...x... \mid x\in U \}\]

We use the symbol $\in$ as ``is an element of'' to indicate membership in a set.\\


{\bf Example sets}: For each of the following, identify whether it's defined using the roster method
or set builder notation and give an example element.
\begin{itemize}
    \item[]$\{ -1, 1\}$\\
    \item[]$\{0, 0 \}$\\
    \item[]$\{-1, 0, 1 \}$\\
    \item[]$\{(x,x,x) \mid x \in \{-1,0,1\} \}$\\
    \item[]$\{ \}$\\
    \item[]$\{ x \in \mathbb{Z} \mid x \geq 0 \}$\\
    \item[]$\{ x \in \mathbb{Z}  \mid x > 0 \}$\\
    \item[]$\{\A,\C,\U,\G\}$ \\
    \item[]$\{\A\U\G, \U\A\G, \U\G\A, \U\A\A \}$\\
\end{itemize}
 \vfill
\section*{Defining functions ratings}


Recall our representation of Netflix users' ratings of movies as $n$-tuples, where
$n$ is the number of movies in the database. 
Each component of the $n$-tuple is $-1$ (didn't like the movie), $0$ 
(neutral rating or didn't watch the movie), or $1$ (liked the movie).

Consider the ratings $P_1 = (-1, 0, 1)$, $P_2 = (1, 1, -1)$, $P_3 = (1, 1, 1)$,
$P_4 = (0,-1,1)$


Which of $P_1$, $P_2$, $P_3$ has movie preferences most similar to $P_4$?

One approach to answer this question: use {\bf functions} to define distance between user preferences.

For example, consider the function 
$d_0: \phantom{the Cartesian product of the set of ratings on 3 movies with itself} \to \phantom{\mathbb{R}}$
given by
\[
d_0 (~(~ (x_1, x_2, x_3), (y_1, y_2, y_3) ~) ~) = \sqrt{ (x_1 - y_1)^2 + (x_2 - y_2)^2 + (x_3 -y_3)^2}
\]


\vfill
\vfill


{\it Extra example:} A new movie is released, and $P_1$ and $P_2$ watch it before $P_3$, and give it
ratings; $P_1$ gives \cmark~and $P_2$ gives \xmark.
Should this movie be recommended to $P_3$? Why or why not?

{\it Extra example:} Define a new function that could be used to compare the $4$-tuples of ratings encoding
movie preferences now that there are four movies in the database.

\vfill
\newpage \vfill
\end{document}
\input{../activity-snippets/defining-sets.tex}
\input{../activity-snippets/rna-motivation.tex}
\documentclass[12pt, oneside]{article}

\usepackage[letterpaper, scale=0.89, centering]{geometry}
\usepackage{fancyhdr}
\setlength{\parindent}{0em}
\setlength{\parskip}{1em}

\pagestyle{fancy}
\fancyhf{}
\renewcommand{\headrulewidth}{0pt}
\rfoot{\href{https://creativecommons.org/licenses/by-nc-sa/2.0/}{CC BY-NC-SA 2.0} Version \today~(\thepage)}

\usepackage{amssymb,amsmath,pifont,amsfonts,comment,enumerate,enumitem}
\usepackage{currfile,xstring,hyperref,tabularx,graphicx,wasysym}
\usepackage[labelformat=empty]{caption}
\usepackage[dvipsnames,table]{xcolor}
\usepackage{multicol,multirow,array,listings,tabularx,lastpage,textcomp,booktabs}

% NOTE(joe): This environment is credit @pnpo (https://tex.stackexchange.com/a/218450)
\lstnewenvironment{algorithm}[1][] %defines the algorithm listing environment
{   
    \lstset{ %this is the stype
        mathescape=true,
        frame=tB,
        numbers=left, 
        numberstyle=\tiny,
        basicstyle=\rmfamily\scriptsize, 
        keywordstyle=\color{black}\bfseries,
        keywords={,procedure, div, for, to, input, output, return, datatype, function, in, if, else, foreach, while, begin, end, }
        numbers=left,
        xleftmargin=.04\textwidth,
        #1
    }
}
{}
\lstnewenvironment{java}[1][]
{   
    \lstset{
        language=java,
        mathescape=true,
        frame=tB,
        numbers=left, 
        numberstyle=\tiny,
        basicstyle=\ttfamily\scriptsize, 
        keywordstyle=\color{black}\bfseries,
        keywords={, int, double, for, return, if, else, while, }
        numbers=left,
        xleftmargin=.04\textwidth,
        #1
    }
}
{}

\newcommand\abs[1]{\lvert~#1~\rvert}
\newcommand{\st}{\mid}

\newcommand{\A}[0]{\texttt{A}}
\newcommand{\C}[0]{\texttt{C}}
\newcommand{\G}[0]{\texttt{G}}
\newcommand{\U}[0]{\texttt{U}}

\newcommand{\cmark}{\ding{51}}
\newcommand{\xmark}{\ding{55}}


\begin{document}
\begin{flushright}
    \StrBefore{\currfilename}{.}
\end{flushright}{\bf New! Recursive Definitions of Sets}: The set $S$ (pick a name) is defined by:
\[
\begin{array}{ll}
\textrm{Basis Step: } & \textrm{Specify finitely many elements of } S\\
\textrm{Recursive Step: } & \textrm{Give rule(s) for creating a new element of } S \textrm{ from known values existing in } S, \\
& \textrm{and potentially other values}. \\
\end{array}
\]
The set $S$ then consists of all and only elements that are put in $S$ by finitely many (a nonnegative integer number) of
applications of the recursive step after the basis step.
\end{document}
%\input{../activity-snippets/set-recursive-definition.tex}
\input{../activity-snippets/set-recursive-examples.tex}
\newpage
\subsection*{Review: Week 1 Monday}
\begin{enumerate}
    \item {\documentclass[12pt, oneside]{article}

\usepackage[letterpaper, scale=0.89, centering]{geometry}
\usepackage{fancyhdr}
\setlength{\parindent}{0em}
\setlength{\parskip}{1em}

\pagestyle{fancy}
\fancyhf{}
\renewcommand{\headrulewidth}{0pt}
\rfoot{\href{https://creativecommons.org/licenses/by-nc-sa/2.0/}{CC BY-NC-SA 2.0} Version \today~(\thepage)}

\usepackage{amssymb,amsmath,pifont,amsfonts,comment,enumerate,enumitem}
\usepackage{currfile,xstring,hyperref,tabularx,graphicx,wasysym}
\usepackage[labelformat=empty]{caption}
\usepackage[dvipsnames,table]{xcolor}
\usepackage{multicol,multirow,array,listings,tabularx,lastpage,textcomp,booktabs}

% NOTE(joe): This environment is credit @pnpo (https://tex.stackexchange.com/a/218450)
\lstnewenvironment{algorithm}[1][] %defines the algorithm listing environment
{   
    \lstset{ %this is the stype
        mathescape=true,
        frame=tB,
        numbers=left, 
        numberstyle=\tiny,
        basicstyle=\rmfamily\scriptsize, 
        keywordstyle=\color{black}\bfseries,
        keywords={,procedure, div, for, to, input, output, return, datatype, function, in, if, else, foreach, while, begin, end, }
        numbers=left,
        xleftmargin=.04\textwidth,
        #1
    }
}
{}
\lstnewenvironment{java}[1][]
{   
    \lstset{
        language=java,
        mathescape=true,
        frame=tB,
        numbers=left, 
        numberstyle=\tiny,
        basicstyle=\ttfamily\scriptsize, 
        keywordstyle=\color{black}\bfseries,
        keywords={, int, double, for, return, if, else, while, }
        numbers=left,
        xleftmargin=.04\textwidth,
        #1
    }
}
{}

\newcommand\abs[1]{\lvert~#1~\rvert}
\newcommand{\st}{\mid}

\newcommand{\A}[0]{\texttt{A}}
\newcommand{\C}[0]{\texttt{C}}
\newcommand{\G}[0]{\texttt{G}}
\newcommand{\U}[0]{\texttt{U}}

\newcommand{\cmark}{\ding{51}}
\newcommand{\xmark}{\ding{55}}


\begin{document}
\begin{flushright}
    \StrBefore{\currfilename}{.}
\end{flushright}Colors can be described as amounts of red, green, and blue mixed together\footnote{This RGB representation
is common in web applications.  Many online tools are available to play around with mixing these colors, 
e.g. \url{https://www.w3schools.com/colors/colors_rgb.asp}. }
Mathematically, a color can be represented as a $3$-tuple $(r, g, b)$ where $r$
represents the red component, $g$ the green component, $b$ the blue component and where each of $r$, $g$, $b$ must
be a value from this collection of numbers:
\begin{quote}
$\{$0, 1, 2, 3, 4, 5, 6, 7, 8, 9, 10, 11, 12, 13, 14, 15, 16, 17, 18, 19, 20, 21, 22, 23, 24, 25, 26, 27, 28, 29, 30, 31, 32, 33, 34, 35, 36, 37, 38, 39, 40, 41, 42, 43, 44, 45, 46, 47, 48, 49, 50, 51, 52, 53, 54, 55, 56, 57, 58, 59, 60, 61, 62, 63, 64, 65, 66, 67, 68, 69, 70, 71, 72, 73, 74, 75, 76, 77, 78, 79, 80, 81, 82, 83, 84, 85, 86, 87, 88, 89, 90, 91, 92, 93, 94, 95, 96, 97, 98, 99, 100, 101, 102, 103, 104, 105, 106, 107, 108, 109, 110, 111, 112, 113, 114, 115, 116, 117, 118, 119, 120, 121, 122, 123, 124, 125, 126, 127, 128, 129, 130, 131, 132, 133, 134, 135, 136, 137, 138, 139, 140, 141, 142, 143, 144, 145, 146, 147, 148, 149, 150, 151, 152, 153, 154, 155, 156, 157, 158, 159, 160, 161, 162, 163, 164, 165, 166, 167, 168, 169, 170, 171, 172, 173, 174, 175, 176, 177, 178, 179, 180, 181, 182, 183, 184, 185, 186, 187, 188, 189, 190, 191, 192, 193, 194, 195, 196, 197, 198, 199, 200, 201, 202, 203, 204, 205, 206, 207, 208, 209, 210, 211, 212, 213, 214, 215, 216, 217, 218, 219, 220, 221, 222, 223, 224, 225, 226, 227, 228, 229, 230, 231, 232, 233, 234, 235, 236, 237, 238, 239, 240, 241, 242, 243, 244, 245, 246, 247, 248, 249, 250, 251, 252, 253, 254, 255$\}$
\end{quote}
\begin{enumerate}
\item \textbf{True} or \textbf{False}: $(1, 3, 4)$ fits the definition of a color above.
\item \textbf{True} or \textbf{False}: $(1, 100, 200, 0)$ fits the definition of a color above.
\item \textbf{True} or \textbf{False}: $(510, 255)$ fits the definition of a color above.
\item \textbf{True} or \textbf{False}: There is a color $(r_1, g_1, b_1)$ where $r_1 + g_1 + b_1$ is greater than $765$.
\item \textbf{True} or \textbf{False}: There is a color $(r_2, g_2, b_2)$ where $r_2 + g_2 + b_2$ is equal to $1$.
\item \textbf{True} or \textbf{False}: Another way to write the collection of allowed values for red, green, and blue components is $$\{x \in \mathbb{N}\mid 0 \leq x \leq 255 \}$$.
\item \textbf{True} or \textbf{False}: Another way to write the collection of allowed values for red, green, and blue components is $$\{n \in \mathbb{Z}\mid 0 \leq n \leq 255 \}$$.
\item \textbf{True} or \textbf{False}: Another way to write the collection of allowed values for red, green, and blue components is $$\{y \in \mathbb{Z}\mid -1 < y \leq 255 \}$$.
\end{enumerate}
\vfill
\end{document}}
    \item {\input{../activity-snippets/quiz-set-membership.tex}}
    \item {\documentclass[12pt, oneside]{article}

\usepackage[letterpaper, scale=0.89, centering]{geometry}
\usepackage{fancyhdr}
\setlength{\parindent}{0em}
\setlength{\parskip}{1em}

\pagestyle{fancy}
\fancyhf{}
\renewcommand{\headrulewidth}{0pt}
\rfoot{\href{https://creativecommons.org/licenses/by-nc-sa/2.0/}{CC BY-NC-SA 2.0} Version \today~(\thepage)}

\usepackage{amssymb,amsmath,pifont,amsfonts,comment,enumerate,enumitem}
\usepackage{currfile,xstring,hyperref,tabularx,graphicx,wasysym}
\usepackage[labelformat=empty]{caption}
\usepackage[dvipsnames,table]{xcolor}
\usepackage{multicol,multirow,array,listings,tabularx,lastpage,textcomp,booktabs}

% NOTE(joe): This environment is credit @pnpo (https://tex.stackexchange.com/a/218450)
\lstnewenvironment{algorithm}[1][] %defines the algorithm listing environment
{   
    \lstset{ %this is the stype
        mathescape=true,
        frame=tB,
        numbers=left, 
        numberstyle=\tiny,
        basicstyle=\rmfamily\scriptsize, 
        keywordstyle=\color{black}\bfseries,
        keywords={,procedure, div, for, to, input, output, return, datatype, function, in, if, else, foreach, while, begin, end, }
        numbers=left,
        xleftmargin=.04\textwidth,
        #1
    }
}
{}
\lstnewenvironment{java}[1][]
{   
    \lstset{
        language=java,
        mathescape=true,
        frame=tB,
        numbers=left, 
        numberstyle=\tiny,
        basicstyle=\ttfamily\scriptsize, 
        keywordstyle=\color{black}\bfseries,
        keywords={, int, double, for, return, if, else, while, }
        numbers=left,
        xleftmargin=.04\textwidth,
        #1
    }
}
{}

\newcommand\abs[1]{\lvert~#1~\rvert}
\newcommand{\st}{\mid}

\newcommand{\A}[0]{\texttt{A}}
\newcommand{\C}[0]{\texttt{C}}
\newcommand{\G}[0]{\texttt{G}}
\newcommand{\U}[0]{\texttt{U}}

\newcommand{\cmark}{\ding{51}}
\newcommand{\xmark}{\ding{55}}


\begin{document}
\begin{flushright}
    \StrBefore{\currfilename}{.}
\end{flushright}Which of the following are (recursive) definitions of the set of integers $\mathbb{Z}$?
(Select True/False for each one.)
\begin{enumerate}
\item 
\[
\begin{array}{ll}
\textrm{Basis Step: } & 5 \in \mathbb{Z} \\
\textrm{Recursive Step: } & \textrm{If } x \in \mathbb{Z} \textrm{, then } x+1 \in \mathbb{Z}
\textrm{ and } x-1 \in \mathbb{Z}
\end{array}
\]
\item 
\[
\begin{array}{ll}
\textrm{Basis Step: } & 0 \in \mathbb{Z} \\
\textrm{Recursive Step: } & \textrm{If } x \in \mathbb{Z} \textrm{, then } x+1 \in \mathbb{Z}
\textrm{ and } x-1 \in \mathbb{Z} \textrm{ and } x+2 \in \mathbb{Z} \textrm{ and } x-2 \in \mathbb{Z}
\end{array}
\]
\item 
\[
\begin{array}{ll}
\textrm{Basis Step: } & 0 \in \mathbb{Z} \\
\textrm{Recursive Step: } & \textrm{If } x \in \mathbb{Z} \textrm{, then } x+2 \in \mathbb{Z}
\textrm{ and } x-1 \in \mathbb{Z}
\end{array}
\]
\item 
\[
\begin{array}{ll}
\textrm{Basis Step: } & 0 \in \mathbb{Z} \\
\textrm{Recursive Step: } & \textrm{If } x \in \mathbb{Z} \textrm{, then } x+1 \in \mathbb{Z}
\textrm{ and } x+2 \in \mathbb{Z}
\end{array}
\]
\end{enumerate}
\end{document}}
\end{enumerate}
\newpage
\section*{Wednesday September 29}
\input{../activity-snippets/set-operations.tex}
\input{../activity-snippets/defining-functions.tex}
\input{../activity-snippets/defining-functions-ratings.tex}
\input{../activity-snippets/defining-functions-recursively.tex}
\newpage
\subsection*{Review: Week 1 Wednesday}
\begin{enumerate}
    \item {\input{../activity-snippets/quiz-ratings-set-operations.tex}}
    \item {\input{../activity-snippets/quiz-defining-functions-domain-codomain.tex}}
    \item {\input{../activity-snippets/quiz-defining-functions-recursively.tex}}
\end{enumerate}
\newpage
\section*{Friday October 1 (Zoom)}

Today's session is on Zoom, log in with your @ucsd.edu account \url{https://ucsd.zoom.us/j/97431852722} Meeting ID: 974 3185 2722

\input{../activity-snippets/why-represent-numbers.tex}
\documentclass[12pt, oneside]{article}

\usepackage[letterpaper, scale=0.89, centering]{geometry}
\usepackage{fancyhdr}
\setlength{\parindent}{0em}
\setlength{\parskip}{1em}

\pagestyle{fancy}
\fancyhf{}
\renewcommand{\headrulewidth}{0pt}
\rfoot{\href{https://creativecommons.org/licenses/by-nc-sa/2.0/}{CC BY-NC-SA 2.0} Version \today~(\thepage)}

\usepackage{amssymb,amsmath,pifont,amsfonts,comment,enumerate,enumitem}
\usepackage{currfile,xstring,hyperref,tabularx,graphicx,wasysym}
\usepackage[labelformat=empty]{caption}
\usepackage[dvipsnames,table]{xcolor}
\usepackage{multicol,multirow,array,listings,tabularx,lastpage,textcomp,booktabs}

% NOTE(joe): This environment is credit @pnpo (https://tex.stackexchange.com/a/218450)
\lstnewenvironment{algorithm}[1][] %defines the algorithm listing environment
{   
    \lstset{ %this is the stype
        mathescape=true,
        frame=tB,
        numbers=left, 
        numberstyle=\tiny,
        basicstyle=\rmfamily\scriptsize, 
        keywordstyle=\color{black}\bfseries,
        keywords={,procedure, div, for, to, input, output, return, datatype, function, in, if, else, foreach, while, begin, end, }
        numbers=left,
        xleftmargin=.04\textwidth,
        #1
    }
}
{}
\lstnewenvironment{java}[1][]
{   
    \lstset{
        language=java,
        mathescape=true,
        frame=tB,
        numbers=left, 
        numberstyle=\tiny,
        basicstyle=\ttfamily\scriptsize, 
        keywordstyle=\color{black}\bfseries,
        keywords={, int, double, for, return, if, else, while, }
        numbers=left,
        xleftmargin=.04\textwidth,
        #1
    }
}
{}

\newcommand\abs[1]{\lvert~#1~\rvert}
\newcommand{\st}{\mid}

\newcommand{\A}[0]{\texttt{A}}
\newcommand{\C}[0]{\texttt{C}}
\newcommand{\G}[0]{\texttt{G}}
\newcommand{\U}[0]{\texttt{U}}

\newcommand{\cmark}{\ding{51}}
\newcommand{\xmark}{\ding{55}}


\begin{document}
\begin{flushright}
    \StrBefore{\currfilename}{.}
\end{flushright}{\bf Definition} For $b$ an integer greater than $1$ and $n$ a positive integer, 
the {\bf base $b$ expansion of $n$}  is
\[
(a_{k-1} \cdots a_1 a_0)_b
\]
where $k$ is a positive integer, $a_0, a_1, \ldots, a_{k-1}$ 
are nonnegative integers less than $b$, $a_{k-1} \neq  0$, and
\[
n =  \sum_{i=0}^{k-1} a_{i} b^{i}
\]
Notice: {\it The base $b$ expansion of a positive integer $n$ is a string over the alphabet 
$\{x \in \mathbb{N} \st x < b\}$
whose leftmost character is nonzero.}
\begin{center}
\begin{tabular}{|c|c|}
\hline
Base $b$ & Collection of possible coefficients in base $b$ expansion of  a positive integer \\
\hline
& \\
Binary ($b=2$) & $\{0,1\}$ \\
\hline
& \\
Ternary ($b=3$) & $\{0,1, 2\}$ \\
\hline
& \\
Octal ($b=8$) & $\{0,1, 2, 3, 4, 5, 6, 7\}$\\
\hline
& \\
Decimal ($b=10$) & $\{0,1, 2, 3, 4, 5, 6, 7, 8, 9\}$\\
\hline
& \\
Hexadecimal ($b=16$) &  $\{0,1, 2, 3, 4, 5, 6, 7, 8, 9, A, B, C, D, E, F\}$\\
& letter coefficient symbols represent numerical values $(A)_{16} = (10)_{10}$\\
&$(B)_{16} = (11)_{10} ~~(C)_{16} = (12)_{10} ~~
 (D)_{16} = (13)_{10} ~~ (E)_{16} = (14)_{10} ~~ (F)_{16} = (15)_{10} $\\
\hline
\end{tabular}
\end{center}

\end{document}
\vfill
\newpage
\input{../activity-snippets/base-expansion-examples.tex}
\documentclass[12pt, oneside]{article}

\usepackage[letterpaper, scale=0.89, centering]{geometry}
\usepackage{fancyhdr}
\setlength{\parindent}{0em}
\setlength{\parskip}{1em}

\pagestyle{fancy}
\fancyhf{}
\renewcommand{\headrulewidth}{0pt}
\rfoot{\href{https://creativecommons.org/licenses/by-nc-sa/2.0/}{CC BY-NC-SA 2.0} Version \today~(\thepage)}

\usepackage{amssymb,amsmath,pifont,amsfonts,comment,enumerate,enumitem}
\usepackage{currfile,xstring,hyperref,tabularx,graphicx,wasysym}
\usepackage[labelformat=empty]{caption}
\usepackage[dvipsnames,table]{xcolor}
\usepackage{multicol,multirow,array,listings,tabularx,lastpage,textcomp,booktabs}

% NOTE(joe): This environment is credit @pnpo (https://tex.stackexchange.com/a/218450)
\lstnewenvironment{algorithm}[1][] %defines the algorithm listing environment
{   
    \lstset{ %this is the stype
        mathescape=true,
        frame=tB,
        numbers=left, 
        numberstyle=\tiny,
        basicstyle=\rmfamily\scriptsize, 
        keywordstyle=\color{black}\bfseries,
        keywords={,procedure, div, for, to, input, output, return, datatype, function, in, if, else, foreach, while, begin, end, }
        numbers=left,
        xleftmargin=.04\textwidth,
        #1
    }
}
{}
\lstnewenvironment{java}[1][]
{   
    \lstset{
        language=java,
        mathescape=true,
        frame=tB,
        numbers=left, 
        numberstyle=\tiny,
        basicstyle=\ttfamily\scriptsize, 
        keywordstyle=\color{black}\bfseries,
        keywords={, int, double, for, return, if, else, while, }
        numbers=left,
        xleftmargin=.04\textwidth,
        #1
    }
}
{}

\newcommand\abs[1]{\lvert~#1~\rvert}
\newcommand{\st}{\mid}

\newcommand{\A}[0]{\texttt{A}}
\newcommand{\C}[0]{\texttt{C}}
\newcommand{\G}[0]{\texttt{G}}
\newcommand{\U}[0]{\texttt{U}}

\newcommand{\cmark}{\ding{51}}
\newcommand{\xmark}{\ding{55}}


\begin{document}
\begin{flushright}
    \StrBefore{\currfilename}{.}
\end{flushright}\fbox{\parbox{\textwidth}{%
{\bf New!} An algorithm is a finite sequence of precise instructions for solving a problem.
\hfill
}}
\end{document}
\input{../activity-snippets/algorithm-half.tex}
\input{../activity-snippets/algorithm-log.tex}
\newpage
\input{../activity-snippets/division-algorithm.tex}
\input{../activity-snippets/base-expansion-algorithms.tex}
\newpage
\subsection*{Review: Week 1 Friday}
\begin{enumerate}
    \item {\input{../activity-snippets/quiz-division-algorithm-calculations.tex}}
    \item {\input{../activity-snippets/quiz-base-expansion-calculations.tex}}
\end{enumerate}
\newpage

\section*{Monday November 29}
\input{../activity-snippets/equivalence-relations-partitions.tex}

Last time, we saw that partitions associated to equivalence relations
were useful in the context of modular arithmetic.
Today we'll look at a different application.

\input{../activity-snippets/equivalence-relations-examples-ratings.tex}
\newpage
\section*{Review}
\begin{enumerate}
    \item Select all and only the partitions of $\{1,2,3,4,5\}$ from the sets below.
    \begin{enumerate}
    \item $\{1,2,3,4,5\}$
    \item $\{\{1,2,3,4,5\}\}$
    \item $\{\{1\},\{2\},\{3\},\{4\},\{5\}\}$
    \item $\{ \{1\}, \{2,3\}, \{4\} \}$
    \item $\{ \{\emptyset, 1, 2\}, \{3,4,5\}\}$
    \end{enumerate}
    \item \hspace{1in}\\ \input{../activity-snippets/quiz-equivalence-class-types.tex}
\end{enumerate}
\newpage

\section*{Wednesday December 1}
\input{../activity-snippets/netflix-clustering-scenario.tex}
\newpage

\section*{Looking forward}

\subsection*{Tips for future classes from the CSE 20 TAs and tutors}
\begin{itemize}
\item In class
\begin{itemize}
\item Go to class
\item Show up to class early because sometimes seats get taken/ the classroom gets full and then you have to sit on the floor
\item There's usually a space for skateboards/longboards/eboards to go at the front or rear of the lecture hall 
\item If you have a flask water bottle please ensure that its secured during a lecture and it cannot fall - putting on the floor often leads to it falling since people sometimes cross your seats.
\item Take notes - it's much faster and more effective to note-take in class than watch recordings after, particularly if you do so longhand
\item Resist the urge to sit in the back. You will be able to focus much better sitting near the front, where there are fewer screens in front of you to distract from the lecture content
\item If you bring your laptop to class to take notes / access class materials, sit towards the back of the room to minimize distractions for people sitting behind you!
\item On zoom it's easy to just type a question out in chat, it might be a little more awkward to do so in person, but it is definitely worth it. Don't feel like you should already know what's being covered
\item Always check you have your iclicker\footnote{iclickers are used in many classes to encourage active participation in class. They're remotes that allow you to respond to multiple choice questions and the instructor can show a histogram of responses in real time.} on you. Just keep it in your backpack permanently. That way you can never forget it. 
\item Don't be afraid to talk to the people next to you during group discussions. Odds are they're as nervous as you are, and you can all benefit from sharing your thoughts and understanding of the material 
\item Certain classes will podcast the lectures, just like Zoom archives lecture recordings, at podcast.ucsd.edu
\item If they aren't podcasted, and you want to record lectures, ask your professor for consent first
\end{itemize}
\item Office hours, tutor hours, and the CSE building
\begin{itemize}
\item Office hours are a good place to hang out and get work done while being able to ask questions as they come up 
\item Office hour attendance is typically much busier in person (and confined to the space in the room)
\item Get to know the CSE building: CSE B260, basement labs, office hours rooms
\item Know how to get in to the building after-hours
\end{itemize}
\item Libraries and on-campus resources
\begin{itemize}
\item Look up what library floors are for what, how to book rooms: East wing of Geisel is open 24/7 (they might ask to see an ID if you stay late), East Wing of Geisel has chess boards and jigsaw puzzles, study pods on the 8th floor, 
free computers/wifi
\item Know Biomed exists and is usually less crowded
\item Most libraries allow you to borrow whiteboards and markers (also laptops, tablets, microphones, and other cool stuff) for 24 hours
\item Take advantage of Dine with a prof / Coffee with a prof program. It's legit free food / coffee once per quarter. 
\item When planning out your daily schedule, think about where classes are, how much time will they take, are their places to eat nearby and how you can schedule social time with friends to nearby areas 
\item Take into account the distances between classes if they are back to back
\end{itemize} 
\item Final exams
\begin{itemize}
\item What are 8am finals? Basically in-person exams are different
\item Don't forget your university card during exams
\item Blue books for exams (what they are, where to get them) 
\item Seating assignments for exams and go early to make sure you're in the right place (check the exits to make sure you're reading it the right way) 
\item Know where your exam is being held (find it on a map at least a day beforehand). Finals are often in strange places that take a while to find 
\end{itemize}
\end{itemize}

\subsection*{CSE department course numbering system}

{\bf Lower division}

\begin{itemize}
\setlength\itemsep{-2pt}
\item CSE 12, Basic Data Structures and Object-Oriented Programming 
\item CSE 15L (2 units), Software Tools and Technique Laboratory 
\item CSE 20 or Math 15A, Introduction to Discrete Mathematics
\item CSE 21 Mathematics for Algorithms and Systems
\item CSE 30, Computer Organization \& Systems Programming
\end{itemize}

{\bf Upper division}

\begin{itemize}
\setlength\itemsep{-2pt}
\item Advanced Data Structures and Programming: CSE 100
\item Theory and Algorithms: CSE 101, CSE 105
\item Software Engineering: CSE 110, CSE 112
\item Systems/Networks: CSE 120 or CSE 123 or CSE 124
\item Programming Languages /Databases: CSE 130 or CSE 132A
\item Security/Cryptography: CSE 107 or CSE 127
\item AI / Machine Learning/ Vision/ Graphics:
CSE 150A or CSE 151A or CSE 151B or CSE 152A or CSE 158 or CSE 167
\item Hardware / Architecture: CSE 140/ CSE 140L Components and Design Techniques for Digital Systems Architecture, CSE 141 / CSE 141L Introduction to Computer Architecture and CSE 141L Project in Computer Architecture (2 units), CSE 142 / CSE 142L Comp Arch Software Perspective
\end{itemize}

\subsection*{Review}
\begin{enumerate}
    \item \hspace{1in}\\ \input{../activity-snippets/quiz-binary-relation-ratings.tex}
    \item \hspace{1in}\\ \input{../activity-snippets/quiz-clustering-ratings.tex} 
\end{enumerate}

\newpage
\section*{Friday December 3}
\input{../activity-snippets/base-expansion-final-review.tex}
\newpage
%! app: TODOapp
%! outcome: TODOoutcome

The bases of RNA strands are elements of the set $B = \{\A, \C, \G, \U \}$. 
The set of RNA strands $S$ is defined (recursively) by:
\[
\begin{array}{ll}
\textrm{Basis Step: } & \A \in S, \C \in S, \U \in S, \G \in S \\
\textrm{Recursive Step: } & \textrm{If } s \in S\textrm{ and }b \in B \textrm{, then }sb \in S
\end{array}
\]
where $sb$ is string concatenation.

Each of the sets below is described using set builder notation. Rewrite them using the roster method. 
\begin{itemize}
\item $\{s \in S ~|~ \text{the leftmost base in $s$ is the same as the rightmost base in $s$ and 
$s$ has length $3$} \}$ 

\vspace{50pt}

\item $\{s \in S ~|~ \text{there are twice as many $\A$s as $\C$s in $s$ and $s$ has length $1$} \}$ 

\vspace{50pt}

\end{itemize}

Certain 
 sequences of bases serve important biological functions in translating RNA to proteins. The following
 recursive definition gives a special set of RNA strands: The set of RNA strands $\hat{S}$ is defined (recursively)
 by 
 
 \begin{alignat*}{2}
\text{Basis step:} & & \A\U\G \in \hat{S}\\
\text{Recursive step:} & \qquad& \text{If } s \in \hat{S} \text{ and } x \in R \text{, then } sx\in \hat{S}\\
 \end{alignat*}
 where $R = \{ \U\U\U, \C\U\C, \A\U\C, \A\U\G, \G\U\U, \C\C\U, \G\C\U, \U\G\G, \G\G\A \}$.

Each of the sets below is described using set builder notation. Rewrite them using the roster method. 
\begin{itemize}
\item $\{s \in \hat{S} ~|~ s \text{ has length less than or equal to $5$} \}$ 

\vspace{50pt}


\item $\{s \in S ~|~ \text{there are twice as many $\C$s as $\A$s in $s$ and $s$ has length $6$} \}$ 

\vspace{50pt}

\end{itemize}
\newpage
%! app: TODOapp
%! outcome: TODOoutcome

Let $W = \mathcal{P}( \{ 1,2,3,4,5\})$. Consider the statement 
\[
\forall A \in W~ \forall B \in W ~ \forall C \in W~ ((A \cap B = A \cap C) \to (B=C) )
\]
Translate the statement to English.
Negate the statement 
and translate this negation to English.
Decide whether the original statement or its negation is true
and justify your decision.

\newpage
\input{../activity-snippets/function-properties-final-review.tex}
\newpage
\input{../activity-snippets/cardinality-final-review.tex}
\newpage
\input{../activity-snippets/modular-arithmetic-final-review.tex}
\newpage
\subsection*{Review}
\begin{enumerate}
    \item \hspace{1in}\\  Please complete the CAPE and TA evaluations.  Once 
    you have done so complete the custom feedback form for this quarter: 
    \url{https://forms.gle/pbYWSRDP2znkciM46}
    
    Then, (we're using the honor system here), write out the statement
    ``I have completed the end  of quarter evaluations"  and you'll receive credit 
    for  this  question.
\end{enumerate}
\newpage

% Today's session is on Zoom, log in with your @ucsd.edu account \url{https://ucsd.zoom.us/j/97431852722} Meeting ID: 974 3185 2722

\section*{Monday October 4}
\input{../activity-snippets/base-expansion-review.tex}
\input{../activity-snippets/base-conversion-algorithm.tex}
\documentclass[12pt, oneside]{article}

\usepackage[letterpaper, scale=0.89, centering]{geometry}
\usepackage{fancyhdr}
\setlength{\parindent}{0em}
\setlength{\parskip}{1em}

\pagestyle{fancy}
\fancyhf{}
\renewcommand{\headrulewidth}{0pt}
\rfoot{\href{https://creativecommons.org/licenses/by-nc-sa/2.0/}{CC BY-NC-SA 2.0} Version \today~(\thepage)}

\usepackage{amssymb,amsmath,pifont,amsfonts,comment,enumerate,enumitem}
\usepackage{currfile,xstring,hyperref,tabularx,graphicx,wasysym}
\usepackage[labelformat=empty]{caption}
\usepackage[dvipsnames,table]{xcolor}
\usepackage{multicol,multirow,array,listings,tabularx,lastpage,textcomp,booktabs}

% NOTE(joe): This environment is credit @pnpo (https://tex.stackexchange.com/a/218450)
\lstnewenvironment{algorithm}[1][] %defines the algorithm listing environment
{   
    \lstset{ %this is the stype
        mathescape=true,
        frame=tB,
        numbers=left, 
        numberstyle=\tiny,
        basicstyle=\rmfamily\scriptsize, 
        keywordstyle=\color{black}\bfseries,
        keywords={,procedure, div, for, to, input, output, return, datatype, function, in, if, else, foreach, while, begin, end, }
        numbers=left,
        xleftmargin=.04\textwidth,
        #1
    }
}
{}
\lstnewenvironment{java}[1][]
{   
    \lstset{
        language=java,
        mathescape=true,
        frame=tB,
        numbers=left, 
        numberstyle=\tiny,
        basicstyle=\ttfamily\scriptsize, 
        keywordstyle=\color{black}\bfseries,
        keywords={, int, double, for, return, if, else, while, }
        numbers=left,
        xleftmargin=.04\textwidth,
        #1
    }
}
{}

\newcommand\abs[1]{\lvert~#1~\rvert}
\newcommand{\st}{\mid}

\newcommand{\A}[0]{\texttt{A}}
\newcommand{\C}[0]{\texttt{C}}
\newcommand{\G}[0]{\texttt{G}}
\newcommand{\U}[0]{\texttt{U}}

\newcommand{\cmark}{\ding{51}}
\newcommand{\xmark}{\ding{55}}


\begin{document}
\begin{flushright}
    \StrBefore{\currfilename}{.}
\end{flushright}{\bf Definition} For $b$ an integer greater than $1$, $w$ a positive integer, 
and $n$ a nonnegative integer
$\underline{\phantom{\hspace{1in}}}$, ~
the {\bf base $b$ fixed-width $w$ expansion of $n$}  is
\[
(a_{w-1} \cdots a_1 a_0)_{b,w}
\]
where  $a_0, a_1, \ldots, a_{w-1}$ are nonnegative integers less than $b$ and
\[
n =  \sum_{i=0}^{w-1} a_{i} b^{i}
\]

\end{document}
\input{../activity-snippets/fixed-width-example.tex}
\documentclass[12pt, oneside]{article}

\usepackage[letterpaper, scale=0.89, centering]{geometry}
\usepackage{fancyhdr}
\setlength{\parindent}{0em}
\setlength{\parskip}{1em}

\pagestyle{fancy}
\fancyhf{}
\renewcommand{\headrulewidth}{0pt}
\rfoot{\href{https://creativecommons.org/licenses/by-nc-sa/2.0/}{CC BY-NC-SA 2.0} Version \today~(\thepage)}

\usepackage{amssymb,amsmath,pifont,amsfonts,comment,enumerate,enumitem}
\usepackage{currfile,xstring,hyperref,tabularx,graphicx,wasysym}
\usepackage[labelformat=empty]{caption}
\usepackage[dvipsnames,table]{xcolor}
\usepackage{multicol,multirow,array,listings,tabularx,lastpage,textcomp,booktabs}

% NOTE(joe): This environment is credit @pnpo (https://tex.stackexchange.com/a/218450)
\lstnewenvironment{algorithm}[1][] %defines the algorithm listing environment
{   
    \lstset{ %this is the stype
        mathescape=true,
        frame=tB,
        numbers=left, 
        numberstyle=\tiny,
        basicstyle=\rmfamily\scriptsize, 
        keywordstyle=\color{black}\bfseries,
        keywords={,procedure, div, for, to, input, output, return, datatype, function, in, if, else, foreach, while, begin, end, }
        numbers=left,
        xleftmargin=.04\textwidth,
        #1
    }
}
{}
\lstnewenvironment{java}[1][]
{   
    \lstset{
        language=java,
        mathescape=true,
        frame=tB,
        numbers=left, 
        numberstyle=\tiny,
        basicstyle=\ttfamily\scriptsize, 
        keywordstyle=\color{black}\bfseries,
        keywords={, int, double, for, return, if, else, while, }
        numbers=left,
        xleftmargin=.04\textwidth,
        #1
    }
}
{}

\newcommand\abs[1]{\lvert~#1~\rvert}
\newcommand{\st}{\mid}

\newcommand{\A}[0]{\texttt{A}}
\newcommand{\C}[0]{\texttt{C}}
\newcommand{\G}[0]{\texttt{G}}
\newcommand{\U}[0]{\texttt{U}}

\newcommand{\cmark}{\ding{51}}
\newcommand{\xmark}{\ding{55}}


\begin{document}
\begin{flushright}
    \StrBefore{\currfilename}{.}
\end{flushright}{\bf Definition} For $b$ an integer greater than $1$, $w$ a positive integer, 
$w'$ a positive  integer, and $x$ a real number the {\bf base $b$ fixed-width 
expansion of $x$ with integer part width $w$  and fractional part width $w'$} is
$(a_{w-1} \cdots a_1 a_0 .  c_{1} \cdots c_{w'})_{b,w,w'}$
where  $a_0, a_1, \ldots, a_{w-1}, c_1, \ldots, c_{w'}$ are nonnegative integers less than $b$ and
$$x \geq \sum_{i=0}^{w-1} a_{i} b^{i} + \sum_{j=1}^{w'}  c_{j} b^{-j} \hfill
\textrm{\qquad and \qquad}
\hfill x < \sum_{i=0}^{w-1} a_{i} b^{i} + \sum_{j=1}^{w'} c_{j} b^{-j} + b^{-w'}$$
\begin{center}
\begin{tabular}{|c|p{5in}|}
\hline
& \\
$3.75$  in fixed-width binary,& \\
integer part width $2$,&\\
 fractional part width $8$ & \\
& \\
\hline
& \\
$0.1$  in fixed-width binary, & \\
integer part width $2$, &\\
 fractional part width $8$ & \\
& \\
\hline
\end{tabular}
\end{center}
\includegraphics[width=2in]{../../resources/images/ArithmeticDemo.png}
Note: Java uses floating point, not fixed width representation, but similar rounding errors appear in both.

\end{document}

\newpage
\subsection*{Review: Week 2 Monday}
\begin{enumerate}
    \item {\input{../activity-snippets/quiz-fixed-width-expansions.tex}}
    \item {\input{../activity-snippets/quiz-expansions-properties.tex}}
\end{enumerate}
\newpage
\section*{Wednesday October 6}
\input{../activity-snippets/expansion-summary.tex}
\input{../activity-snippets/negative-int-expansions.tex}
\newpage
\input{../activity-snippets/calculating-2s-complement.tex}
\input{../activity-snippets/representing-zero.tex}
\newpage
\input{../activity-snippets/fixed-width-addition.tex}
\newpage
\subsection*{Review: Week 2 Wednesday}
\begin{enumerate}
    \item {\input{../activity-snippets/quiz-representing-negatives.tex}}
    \item {\input{../activity-snippets/quiz-fixed-width-addition.tex}}
\end{enumerate}
\newpage
\section*{Friday October 8}
\input{../activity-snippets/circuits-basics.tex}
\documentclass[12pt, oneside]{article}

\usepackage[letterpaper, scale=0.89, centering]{geometry}
\usepackage{fancyhdr}
\setlength{\parindent}{0em}
\setlength{\parskip}{1em}

\pagestyle{fancy}
\fancyhf{}
\renewcommand{\headrulewidth}{0pt}
\rfoot{\href{https://creativecommons.org/licenses/by-nc-sa/2.0/}{CC BY-NC-SA 2.0} Version \today~(\thepage)}

\usepackage{amssymb,amsmath,pifont,amsfonts,comment,enumerate,enumitem}
\usepackage{currfile,xstring,hyperref,tabularx,graphicx,wasysym}
\usepackage[labelformat=empty]{caption}
\usepackage[dvipsnames,table]{xcolor}
\usepackage{multicol,multirow,array,listings,tabularx,lastpage,textcomp,booktabs}

% NOTE(joe): This environment is credit @pnpo (https://tex.stackexchange.com/a/218450)
\lstnewenvironment{algorithm}[1][] %defines the algorithm listing environment
{   
    \lstset{ %this is the stype
        mathescape=true,
        frame=tB,
        numbers=left, 
        numberstyle=\tiny,
        basicstyle=\rmfamily\scriptsize, 
        keywordstyle=\color{black}\bfseries,
        keywords={,procedure, div, for, to, input, output, return, datatype, function, in, if, else, foreach, while, begin, end, }
        numbers=left,
        xleftmargin=.04\textwidth,
        #1
    }
}
{}
\lstnewenvironment{java}[1][]
{   
    \lstset{
        language=java,
        mathescape=true,
        frame=tB,
        numbers=left, 
        numberstyle=\tiny,
        basicstyle=\ttfamily\scriptsize, 
        keywordstyle=\color{black}\bfseries,
        keywords={, int, double, for, return, if, else, while, }
        numbers=left,
        xleftmargin=.04\textwidth,
        #1
    }
}
{}

\newcommand\abs[1]{\lvert~#1~\rvert}
\newcommand{\st}{\mid}

\newcommand{\A}[0]{\texttt{A}}
\newcommand{\C}[0]{\texttt{C}}
\newcommand{\G}[0]{\texttt{G}}
\newcommand{\U}[0]{\texttt{U}}

\newcommand{\cmark}{\ding{51}}
\newcommand{\xmark}{\ding{55}}


\begin{document}
\begin{flushright}
    \StrBefore{\currfilename}{.}
\end{flushright}\begin{multicols}{2}
\begin{center}\begin{tabular}{cc|c}
Inputs &  & Output \\
$x$ & $y$ & $x \text{ AND } y$  \\
\hline
$1$ & $1$ & $1$\\
$1$ & $0$ & $0$\\
$0$ & $1$ & $0$\\
$0$ & $0$ & $0$\\
\end{tabular}\end{center}
\columnbreak
\begin{center}\includegraphics[height=0.6in]{../../resources/images/xANDy.png} \end{center}
\end{multicols}
\begin{multicols}{2}
\begin{center}\begin{tabular}{cc|c}
Inputs &  & Output \\
$x$ & $y$ & $x \text{ XOR } y$  \\
\hline
$1$ & $1$ & $0$\\
$1$ & $0$ & $1$\\
$0$ & $1$ & $1$\\
$0$ & $0$ & $0$\\
\end{tabular}\end{center}
\columnbreak
\begin{center}\includegraphics[height=0.4in]{../../resources/images/xXORy.png} \end{center}
\end{multicols}
\begin{multicols}{2}
\begin{center}\begin{tabular}{c|c}
Input  & Output \\
$x$ & $\text{NOT } x$  \\
\hline
$1$ & $0$\\
$0$ & $1$\\
\end{tabular}\end{center}
\columnbreak
\begin{center}\includegraphics[height=0.5in]{../../resources/images/NOTx.png} \end{center}
\end{multicols}
% \begin{center}
%     \begin{tabular}{p{2in}p{2in}p{2in}}
%     \begin{center}\begin{tabular}{cc|c}
%     Inputs &  & Output \\
%     $x$ & $y$ & $x \text{ AND } y$  \\
%     \hline
%     $1$ & $1$ & $1$\\
%     $1$ & $0$ & $0$\\
%     $0$ & $1$ & $0$\\
%     $0$ & $0$ & $0$\\
%     \end{tabular}\end{center}
%     &
%     \begin{center}\begin{tabular}{cc|c}
%     Inputs &  & Output \\
%     $x$ & $y$ & $x \text{ XOR } y$  \\
%     \hline
%     $1$ & $1$ & $0$\\
%     $1$ & $0$ & $1$\\
%     $0$ & $1$ & $1$\\
%     $0$ & $0$ & $0$\\
%     \end{tabular}\end{center}
%     &
%     \begin{center}\begin{tabular}{c|c}
%     Input  & Output \\
%     $x$ & $\text{NOT } x$  \\
%     \hline
%     $1$ & $0$\\
%     $0$ & $1$\\
%     \end{tabular}\end{center}
%     \\
%     \begin{center}\includegraphics[height=0.6in]{../../resources/images/xANDy.png} \end{center} & 
%     \begin{center}\includegraphics[height=0.4in]{../../resources/images/xXORy.png} \end{center}& 
%     \begin{center}\includegraphics[height=0.5in]{../../resources/images/NOTx.png} \end{center}
%     \end{tabular}
%     \end{center}
\end{document}
\input{../activity-snippets/digital-circuits-basic-examples.tex}
\newpage
\input{../activity-snippets/half-adder-circuit.tex}
\newpage
\input{../activity-snippets/two-bit-adder-circuit.tex}
\newpage
\subsection*{Review: Week 2 Friday}
\begin{enumerate}
    \item {\input{../activity-snippets/quiz-circuit-tracing.tex}}
    \item {\input{../activity-snippets/quiz-circuit-implementing-operation.tex}}
\end{enumerate}

\newpage

\section*{Monday October 11}
\input{../activity-snippets/logical-operators.tex}
\input{../activity-snippets/logical-operators-truth-tables.tex}
\input{../activity-snippets/logical-operators-example-truth-table.tex}
\newpage
\input{../activity-snippets/truth-table-to-compound-proposition.tex}
\documentclass[12pt, oneside]{article}

\usepackage[letterpaper, scale=0.89, centering]{geometry}
\usepackage{fancyhdr}
\setlength{\parindent}{0em}
\setlength{\parskip}{1em}

\pagestyle{fancy}
\fancyhf{}
\renewcommand{\headrulewidth}{0pt}
\rfoot{\href{https://creativecommons.org/licenses/by-nc-sa/2.0/}{CC BY-NC-SA 2.0} Version \today~(\thepage)}

\usepackage{amssymb,amsmath,pifont,amsfonts,comment,enumerate,enumitem}
\usepackage{currfile,xstring,hyperref,tabularx,graphicx,wasysym}
\usepackage[labelformat=empty]{caption}
\usepackage[dvipsnames,table]{xcolor}
\usepackage{multicol,multirow,array,listings,tabularx,lastpage,textcomp,booktabs}

% NOTE(joe): This environment is credit @pnpo (https://tex.stackexchange.com/a/218450)
\lstnewenvironment{algorithm}[1][] %defines the algorithm listing environment
{   
    \lstset{ %this is the stype
        mathescape=true,
        frame=tB,
        numbers=left, 
        numberstyle=\tiny,
        basicstyle=\rmfamily\scriptsize, 
        keywordstyle=\color{black}\bfseries,
        keywords={,procedure, div, for, to, input, output, return, datatype, function, in, if, else, foreach, while, begin, end, }
        numbers=left,
        xleftmargin=.04\textwidth,
        #1
    }
}
{}
\lstnewenvironment{java}[1][]
{   
    \lstset{
        language=java,
        mathescape=true,
        frame=tB,
        numbers=left, 
        numberstyle=\tiny,
        basicstyle=\ttfamily\scriptsize, 
        keywordstyle=\color{black}\bfseries,
        keywords={, int, double, for, return, if, else, while, }
        numbers=left,
        xleftmargin=.04\textwidth,
        #1
    }
}
{}

\newcommand\abs[1]{\lvert~#1~\rvert}
\newcommand{\st}{\mid}

\newcommand{\A}[0]{\texttt{A}}
\newcommand{\C}[0]{\texttt{C}}
\newcommand{\G}[0]{\texttt{G}}
\newcommand{\U}[0]{\texttt{U}}

\newcommand{\cmark}{\ding{51}}
\newcommand{\xmark}{\ding{55}}


\begin{document}
\begin{flushright}
    \StrBefore{\currfilename}{.}
\end{flushright}{\bf  Definition} An expression built of variables and logical 
operators is in {\bf disjunctive normal form}  (DNF) means
that it is an OR of ANDs of variables and their negations.
{\bf  Definition} An expression built of variables and logical 
operators is in {\bf conjunctive normal form}  (CNF) means
that it is an AND of ORs of variables and their negations.

\end{document}
\input{../activity-snippets/dnf-cnf-example.tex}
\newpage
\subsection*{Review: Week 3 Monday}
\begin{enumerate}
    \item \input{../activity-snippets/quiz-circuit-tracing-with-or.tex}
    \item \input{../activity-snippets/quiz-cnf-dnf.tex}
\end{enumerate}
\newpage
\section*{Wednesday October 13}
\documentclass[12pt, oneside]{article}

\usepackage[letterpaper, scale=0.89, centering]{geometry}
\usepackage{fancyhdr}
\setlength{\parindent}{0em}
\setlength{\parskip}{1em}

\pagestyle{fancy}
\fancyhf{}
\renewcommand{\headrulewidth}{0pt}
\rfoot{\href{https://creativecommons.org/licenses/by-nc-sa/2.0/}{CC BY-NC-SA 2.0} Version \today~(\thepage)}

\usepackage{amssymb,amsmath,pifont,amsfonts,comment,enumerate,enumitem}
\usepackage{currfile,xstring,hyperref,tabularx,graphicx,wasysym}
\usepackage[labelformat=empty]{caption}
\usepackage[dvipsnames,table]{xcolor}
\usepackage{multicol,multirow,array,listings,tabularx,lastpage,textcomp,booktabs}

% NOTE(joe): This environment is credit @pnpo (https://tex.stackexchange.com/a/218450)
\lstnewenvironment{algorithm}[1][] %defines the algorithm listing environment
{   
    \lstset{ %this is the stype
        mathescape=true,
        frame=tB,
        numbers=left, 
        numberstyle=\tiny,
        basicstyle=\rmfamily\scriptsize, 
        keywordstyle=\color{black}\bfseries,
        keywords={,procedure, div, for, to, input, output, return, datatype, function, in, if, else, foreach, while, begin, end, }
        numbers=left,
        xleftmargin=.04\textwidth,
        #1
    }
}
{}
\lstnewenvironment{java}[1][]
{   
    \lstset{
        language=java,
        mathescape=true,
        frame=tB,
        numbers=left, 
        numberstyle=\tiny,
        basicstyle=\ttfamily\scriptsize, 
        keywordstyle=\color{black}\bfseries,
        keywords={, int, double, for, return, if, else, while, }
        numbers=left,
        xleftmargin=.04\textwidth,
        #1
    }
}
{}

\newcommand\abs[1]{\lvert~#1~\rvert}
\newcommand{\st}{\mid}

\newcommand{\A}[0]{\texttt{A}}
\newcommand{\C}[0]{\texttt{C}}
\newcommand{\G}[0]{\texttt{G}}
\newcommand{\U}[0]{\texttt{U}}

\newcommand{\cmark}{\ding{51}}
\newcommand{\xmark}{\ding{55}}


\begin{document}
\begin{flushright}
    \StrBefore{\currfilename}{.}
\end{flushright}{\bf Proposition}: Declarative sentence that is true or false (not both).
{\bf Propositional variable}: Variable that represents a proposition.
{\bf Compound proposition}: New proposition formed from existing propositions (potentially) using logical operators.
{\it Note}: A propositional variable is one example of a compound proposition.
{\bf Truth table}: Table with one row for each of the possible combinations of truth values of the input and 
    an additional column that shows the truth value of the result of the operation corresponding to a particular row.
    

\end{document}
\input{../activity-snippets/logical-equivalence.tex}
\vfill
\input{../activity-snippets/tautology-contradiction-contingency-examples.tex}
\vfill
\input{../activity-snippets/logical-equivalence-extra-example.tex}
\vfill
\input{../activity-snippets/logical-operators-full-truth-table.tex}
\input{../activity-snippets/hypothesis-conclusion.tex}
\input{../activity-snippets/converse-inverse-contrapositive.tex}
\vfill
\documentclass[12pt, oneside]{article}

\usepackage[letterpaper, scale=0.89, centering]{geometry}
\usepackage{fancyhdr}
\setlength{\parindent}{0em}
\setlength{\parskip}{1em}

\pagestyle{fancy}
\fancyhf{}
\renewcommand{\headrulewidth}{0pt}
\rfoot{\href{https://creativecommons.org/licenses/by-nc-sa/2.0/}{CC BY-NC-SA 2.0} Version \today~(\thepage)}

\usepackage{amssymb,amsmath,pifont,amsfonts,comment,enumerate,enumitem}
\usepackage{currfile,xstring,hyperref,tabularx,graphicx,wasysym}
\usepackage[labelformat=empty]{caption}
\usepackage[dvipsnames,table]{xcolor}
\usepackage{multicol,multirow,array,listings,tabularx,lastpage,textcomp,booktabs}

% NOTE(joe): This environment is credit @pnpo (https://tex.stackexchange.com/a/218450)
\lstnewenvironment{algorithm}[1][] %defines the algorithm listing environment
{   
    \lstset{ %this is the stype
        mathescape=true,
        frame=tB,
        numbers=left, 
        numberstyle=\tiny,
        basicstyle=\rmfamily\scriptsize, 
        keywordstyle=\color{black}\bfseries,
        keywords={,procedure, div, for, to, input, output, return, datatype, function, in, if, else, foreach, while, begin, end, }
        numbers=left,
        xleftmargin=.04\textwidth,
        #1
    }
}
{}
\lstnewenvironment{java}[1][]
{   
    \lstset{
        language=java,
        mathescape=true,
        frame=tB,
        numbers=left, 
        numberstyle=\tiny,
        basicstyle=\ttfamily\scriptsize, 
        keywordstyle=\color{black}\bfseries,
        keywords={, int, double, for, return, if, else, while, }
        numbers=left,
        xleftmargin=.04\textwidth,
        #1
    }
}
{}

\newcommand\abs[1]{\lvert~#1~\rvert}
\newcommand{\st}{\mid}

\newcommand{\A}[0]{\texttt{A}}
\newcommand{\C}[0]{\texttt{C}}
\newcommand{\G}[0]{\texttt{G}}
\newcommand{\U}[0]{\texttt{U}}

\newcommand{\cmark}{\ding{51}}
\newcommand{\xmark}{\ding{55}}


\begin{document}
\begin{flushright}
    \StrBefore{\currfilename}{.}
\end{flushright}We can use a recursive definition to describe all 
compound propositions that use propositional variables 
from a specified collection.  Here's the definition
for all compound propositions whose propositional variables 
are in $\{p, q\}$.
\[
\begin{array}{ll}
\textrm{Basis Step: } & p \textrm{ and } q \textrm{ are each a compound proposition} \\
\textrm{Recursive Step: } & \textrm{If } x \textrm{ is a compound proposition then so is } (\lnot x) 
\textrm{ and if } \\
& x \textrm{ and } y \textrm{ are both compound propositions then so is each of }\\
&(x \land y), (x \oplus y), (x \lor y), (x \to y), (x \leftrightarrow y)
\end{array}
\]
\end{document}
\input{../activity-snippets/compound-propositions-precedence.tex}
\newpage
\input{../activity-snippets/logical-equivalence-identities.tex}
\newpage
\subsection*{Review: Week 3 Wednesday}
\begin{enumerate}
    \item \input{../activity-snippets/quiz-truth-values-or-and.tex}
    \item \input{../activity-snippets/quiz-truth-values-conditional.tex}
\end{enumerate}
\newpage
\section*{Friday October 15}
\input{../activity-snippets/logical-operators-english-synonyms.tex}
\newpage
\newpage
\input{../activity-snippets/compound-propositions-translation.tex}
\newpage
\input{../activity-snippets/consistency-def.tex}
\input{../activity-snippets/consistency-example.tex}
\newpage
\subsection*{Review: Week 3 Friday}
\begin{enumerate}
    \item \input{../activity-snippets/quiz-compound-propositions-translation.tex}
    \item \input{../activity-snippets/quiz-consistency.tex}
\end{enumerate}

\newpage

\section*{Monday October 18}
\input{../activity-snippets/algorithm-redundancy.tex}
\newpage
\documentclass[12pt, oneside]{article}

\usepackage[letterpaper, scale=0.89, centering]{geometry}
\usepackage{fancyhdr}
\setlength{\parindent}{0em}
\setlength{\parskip}{1em}

\pagestyle{fancy}
\fancyhf{}
\renewcommand{\headrulewidth}{0pt}
\rfoot{\href{https://creativecommons.org/licenses/by-nc-sa/2.0/}{CC BY-NC-SA 2.0} Version \today~(\thepage)}

\usepackage{amssymb,amsmath,pifont,amsfonts,comment,enumerate,enumitem}
\usepackage{currfile,xstring,hyperref,tabularx,graphicx,wasysym}
\usepackage[labelformat=empty]{caption}
\usepackage[dvipsnames,table]{xcolor}
\usepackage{multicol,multirow,array,listings,tabularx,lastpage,textcomp,booktabs}

% NOTE(joe): This environment is credit @pnpo (https://tex.stackexchange.com/a/218450)
\lstnewenvironment{algorithm}[1][] %defines the algorithm listing environment
{   
    \lstset{ %this is the stype
        mathescape=true,
        frame=tB,
        numbers=left, 
        numberstyle=\tiny,
        basicstyle=\rmfamily\scriptsize, 
        keywordstyle=\color{black}\bfseries,
        keywords={,procedure, div, for, to, input, output, return, datatype, function, in, if, else, foreach, while, begin, end, }
        numbers=left,
        xleftmargin=.04\textwidth,
        #1
    }
}
{}
\lstnewenvironment{java}[1][]
{   
    \lstset{
        language=java,
        mathescape=true,
        frame=tB,
        numbers=left, 
        numberstyle=\tiny,
        basicstyle=\ttfamily\scriptsize, 
        keywordstyle=\color{black}\bfseries,
        keywords={, int, double, for, return, if, else, while, }
        numbers=left,
        xleftmargin=.04\textwidth,
        #1
    }
}
{}

\newcommand\abs[1]{\lvert~#1~\rvert}
\newcommand{\st}{\mid}

\newcommand{\A}[0]{\texttt{A}}
\newcommand{\C}[0]{\texttt{C}}
\newcommand{\G}[0]{\texttt{G}}
\newcommand{\U}[0]{\texttt{U}}

\newcommand{\cmark}{\ding{51}}
\newcommand{\xmark}{\ding{55}}


\begin{document}
\begin{flushright}
    \StrBefore{\currfilename}{.}
\end{flushright}{\bf Definition}: The {\bf Cartesian product} of the sets $A$ and $B$, 
$A \times B$, is the set of all ordered pairs $(a, b)$, where $a \in A$ and $b \in B$. 
That is: $A \times B = \{(a, b) \mid (a \in A) \land (b \in B)\}$.
The Cartesian product of the sets $A_1, A_2, \ldots ,A_n$, denoted by 
$A_1 \times A_2 \times \cdots \times A_n$, is the
set of ordered n-tuples $(a_1, a_2,...,a_n)$, where $a_i$ belongs to 
$A_i$ for $i = 1, 2,\ldots,n$. That is,
\[
    A_1 \times A_2 \times \cdots \times A_n = \{(a_1, a_2,\ldots,a_n) \mid a_i \in A_i \textrm{ for } i = 1, 2,\ldots,n\}
\]
\end{document}
\input{../activity-snippets/algorithm-rna-mutation-insertion-deletion.tex}
\input{../activity-snippets/rna-mutation-insertion-deletion-example.tex}
\newpage

\newpage
\subsection*{Review}
\begin{enumerate}
\item \hspace{1in}\\ \input{../activity-snippets/quiz-building-circuit.tex}
\item \hspace{1in}\\ \input{../activity-snippets/quiz-redundancy-algorithm.tex}
\item \hspace{1in}\\ \input{../activity-snippets/quiz-rna-mutation-insertion-deletion.tex}
\end{enumerate}

\newpage
\section*{Wednesday October 20}
\documentclass[12pt, oneside]{article}

\usepackage[letterpaper, scale=0.89, centering]{geometry}
\usepackage{fancyhdr}
\setlength{\parindent}{0em}
\setlength{\parskip}{1em}

\pagestyle{fancy}
\fancyhf{}
\renewcommand{\headrulewidth}{0pt}
\rfoot{\href{https://creativecommons.org/licenses/by-nc-sa/2.0/}{CC BY-NC-SA 2.0} Version \today~(\thepage)}

\usepackage{amssymb,amsmath,pifont,amsfonts,comment,enumerate,enumitem}
\usepackage{currfile,xstring,hyperref,tabularx,graphicx,wasysym}
\usepackage[labelformat=empty]{caption}
\usepackage[dvipsnames,table]{xcolor}
\usepackage{multicol,multirow,array,listings,tabularx,lastpage,textcomp,booktabs}

% NOTE(joe): This environment is credit @pnpo (https://tex.stackexchange.com/a/218450)
\lstnewenvironment{algorithm}[1][] %defines the algorithm listing environment
{   
    \lstset{ %this is the stype
        mathescape=true,
        frame=tB,
        numbers=left, 
        numberstyle=\tiny,
        basicstyle=\rmfamily\scriptsize, 
        keywordstyle=\color{black}\bfseries,
        keywords={,procedure, div, for, to, input, output, return, datatype, function, in, if, else, foreach, while, begin, end, }
        numbers=left,
        xleftmargin=.04\textwidth,
        #1
    }
}
{}
\lstnewenvironment{java}[1][]
{   
    \lstset{
        language=java,
        mathescape=true,
        frame=tB,
        numbers=left, 
        numberstyle=\tiny,
        basicstyle=\ttfamily\scriptsize, 
        keywordstyle=\color{black}\bfseries,
        keywords={, int, double, for, return, if, else, while, }
        numbers=left,
        xleftmargin=.04\textwidth,
        #1
    }
}
{}

\newcommand\abs[1]{\lvert~#1~\rvert}
\newcommand{\st}{\mid}

\newcommand{\A}[0]{\texttt{A}}
\newcommand{\C}[0]{\texttt{C}}
\newcommand{\G}[0]{\texttt{G}}
\newcommand{\U}[0]{\texttt{U}}

\newcommand{\cmark}{\ding{51}}
\newcommand{\xmark}{\ding{55}}


\begin{document}
\begin{flushright}
    \StrBefore{\currfilename}{.}
\end{flushright}{\bf  Definition}: A  {\bf predicate}  is  a function from a given set (domain) to $\{T,F\}$.
A predicate can be applied, or {\bf evaluated} at, an element of the domain.
Usually, a predicate {\it describes a  property} that domain elements may or may not have.
Two predicates over the same domain are {\bf equivalent} means they evaluate to
the same truth values for all possible assignments of domain elements to the
input. In other words, they are equivalent means that they are equal as functions.
To define a predicate, we must specify its domain and its value at each domain element.
The rule assigning truth values to domain elements can be specified using a formula, 
English description, in a table (if the domain is finite), or recursively (if the domain is recursively
defined).
\end{document}
\input{../activity-snippets/predicate-examples-finite-domain.tex}
\vfill
\documentclass[12pt, oneside]{article}

\usepackage[letterpaper, scale=0.89, centering]{geometry}
\usepackage{fancyhdr}
\setlength{\parindent}{0em}
\setlength{\parskip}{1em}

\pagestyle{fancy}
\fancyhf{}
\renewcommand{\headrulewidth}{0pt}
\rfoot{\href{https://creativecommons.org/licenses/by-nc-sa/2.0/}{CC BY-NC-SA 2.0} Version \today~(\thepage)}

\usepackage{amssymb,amsmath,pifont,amsfonts,comment,enumerate,enumitem}
\usepackage{currfile,xstring,hyperref,tabularx,graphicx,wasysym}
\usepackage[labelformat=empty]{caption}
\usepackage[dvipsnames,table]{xcolor}
\usepackage{multicol,multirow,array,listings,tabularx,lastpage,textcomp,booktabs}

% NOTE(joe): This environment is credit @pnpo (https://tex.stackexchange.com/a/218450)
\lstnewenvironment{algorithm}[1][] %defines the algorithm listing environment
{   
    \lstset{ %this is the stype
        mathescape=true,
        frame=tB,
        numbers=left, 
        numberstyle=\tiny,
        basicstyle=\rmfamily\scriptsize, 
        keywordstyle=\color{black}\bfseries,
        keywords={,procedure, div, for, to, input, output, return, datatype, function, in, if, else, foreach, while, begin, end, }
        numbers=left,
        xleftmargin=.04\textwidth,
        #1
    }
}
{}
\lstnewenvironment{java}[1][]
{   
    \lstset{
        language=java,
        mathescape=true,
        frame=tB,
        numbers=left, 
        numberstyle=\tiny,
        basicstyle=\ttfamily\scriptsize, 
        keywordstyle=\color{black}\bfseries,
        keywords={, int, double, for, return, if, else, while, }
        numbers=left,
        xleftmargin=.04\textwidth,
        #1
    }
}
{}

\newcommand\abs[1]{\lvert~#1~\rvert}
\newcommand{\st}{\mid}

\newcommand{\A}[0]{\texttt{A}}
\newcommand{\C}[0]{\texttt{C}}
\newcommand{\G}[0]{\texttt{G}}
\newcommand{\U}[0]{\texttt{U}}

\newcommand{\cmark}{\ding{51}}
\newcommand{\xmark}{\ding{55}}


\begin{document}
\begin{flushright}
    \StrBefore{\currfilename}{.}
\end{flushright}{\bf Definition}: The {\bf truth  set} of a  predicate is the collection of all elements in its
domain where the predicate evaluates to $T$.
Notice that specifying the domain and the truth set is sufficient for defining
a predicate.
\end{document}
\input{../activity-snippets/predicate-truth-set-example.tex}
\newpage
\documentclass[12pt, oneside]{article}

\usepackage[letterpaper, scale=0.89, centering]{geometry}
\usepackage{fancyhdr}
\setlength{\parindent}{0em}
\setlength{\parskip}{1em}

\pagestyle{fancy}
\fancyhf{}
\renewcommand{\headrulewidth}{0pt}
\rfoot{\href{https://creativecommons.org/licenses/by-nc-sa/2.0/}{CC BY-NC-SA 2.0} Version \today~(\thepage)}

\usepackage{amssymb,amsmath,pifont,amsfonts,comment,enumerate,enumitem}
\usepackage{currfile,xstring,hyperref,tabularx,graphicx,wasysym}
\usepackage[labelformat=empty]{caption}
\usepackage[dvipsnames,table]{xcolor}
\usepackage{multicol,multirow,array,listings,tabularx,lastpage,textcomp,booktabs}

% NOTE(joe): This environment is credit @pnpo (https://tex.stackexchange.com/a/218450)
\lstnewenvironment{algorithm}[1][] %defines the algorithm listing environment
{   
    \lstset{ %this is the stype
        mathescape=true,
        frame=tB,
        numbers=left, 
        numberstyle=\tiny,
        basicstyle=\rmfamily\scriptsize, 
        keywordstyle=\color{black}\bfseries,
        keywords={,procedure, div, for, to, input, output, return, datatype, function, in, if, else, foreach, while, begin, end, }
        numbers=left,
        xleftmargin=.04\textwidth,
        #1
    }
}
{}
\lstnewenvironment{java}[1][]
{   
    \lstset{
        language=java,
        mathescape=true,
        frame=tB,
        numbers=left, 
        numberstyle=\tiny,
        basicstyle=\ttfamily\scriptsize, 
        keywordstyle=\color{black}\bfseries,
        keywords={, int, double, for, return, if, else, while, }
        numbers=left,
        xleftmargin=.04\textwidth,
        #1
    }
}
{}

\newcommand\abs[1]{\lvert~#1~\rvert}
\newcommand{\st}{\mid}

\newcommand{\A}[0]{\texttt{A}}
\newcommand{\C}[0]{\texttt{C}}
\newcommand{\G}[0]{\texttt{G}}
\newcommand{\U}[0]{\texttt{U}}

\newcommand{\cmark}{\ding{51}}
\newcommand{\xmark}{\ding{55}}


\begin{document}
\begin{flushright}
    \StrBefore{\currfilename}{.}
\end{flushright}The {\bf universal quantification} of predicate $P(x)$ over
domain $U$ is the statement ``$P(x)$ for all values of $x$ in the domain $U$''
and is written $\forall x P(x)$ or $\forall x \in U ~P(x)$. 
When the domain is finite, universal quantification over the domain 
is equivalent to iterated {\it conjunction} (ands).
The {\bf existential quantification} of predicate $P(x)$ 
over domain $U$ is the statement ``There exists an element $x$ 
in the domain $U$ such that $P(x)$'' and is written $\exists x P(x)$
for $\exists x \in U ~P(x)$. 
When the domain is finite, existential quantification over the domain 
is equivalent to iterated {\it disjunction} (ors).
An element for which $P(x) = F$ is called a {\bf counterexample} of $\forall x P(x)$.
An element for which $P(x) = T$ is called a {\bf witness} of $\exists x P(x)$.

\end{document}
\input{../activity-snippets/quantification-logical-equivalence.tex}
\input{../activity-snippets/quantification-examples-finite-domain.tex}
\input{../activity-snippets/predicate-rna-example.tex}
\newpage
\subsection*{Review}
\begin{enumerate}
\item \hspace{1in}\\ \input{../activity-snippets/quiz-predicates-finite-domain.tex}
\item \hspace{1in}\\ \input{../activity-snippets/quiz-predicates.tex}
\item \hspace{1in}\\ \input{../activity-snippets/quiz-predicates-rna.tex}
\end{enumerate}

\newpage
\section*{Friday October 22}
\documentclass[12pt, oneside]{article}

\usepackage[letterpaper, scale=0.89, centering]{geometry}
\usepackage{fancyhdr}
\setlength{\parindent}{0em}
\setlength{\parskip}{1em}

\pagestyle{fancy}
\fancyhf{}
\renewcommand{\headrulewidth}{0pt}
\rfoot{\href{https://creativecommons.org/licenses/by-nc-sa/2.0/}{CC BY-NC-SA 2.0} Version \today~(\thepage)}

\usepackage{amssymb,amsmath,pifont,amsfonts,comment,enumerate,enumitem}
\usepackage{currfile,xstring,hyperref,tabularx,graphicx,wasysym}
\usepackage[labelformat=empty]{caption}
\usepackage[dvipsnames,table]{xcolor}
\usepackage{multicol,multirow,array,listings,tabularx,lastpage,textcomp,booktabs}

% NOTE(joe): This environment is credit @pnpo (https://tex.stackexchange.com/a/218450)
\lstnewenvironment{algorithm}[1][] %defines the algorithm listing environment
{   
    \lstset{ %this is the stype
        mathescape=true,
        frame=tB,
        numbers=left, 
        numberstyle=\tiny,
        basicstyle=\rmfamily\scriptsize, 
        keywordstyle=\color{black}\bfseries,
        keywords={,procedure, div, for, to, input, output, return, datatype, function, in, if, else, foreach, while, begin, end, }
        numbers=left,
        xleftmargin=.04\textwidth,
        #1
    }
}
{}
\lstnewenvironment{java}[1][]
{   
    \lstset{
        language=java,
        mathescape=true,
        frame=tB,
        numbers=left, 
        numberstyle=\tiny,
        basicstyle=\ttfamily\scriptsize, 
        keywordstyle=\color{black}\bfseries,
        keywords={, int, double, for, return, if, else, while, }
        numbers=left,
        xleftmargin=.04\textwidth,
        #1
    }
}
{}

\newcommand\abs[1]{\lvert~#1~\rvert}
\newcommand{\st}{\mid}

\newcommand{\A}[0]{\texttt{A}}
\newcommand{\C}[0]{\texttt{C}}
\newcommand{\G}[0]{\texttt{G}}
\newcommand{\U}[0]{\texttt{U}}

\newcommand{\cmark}{\ding{51}}
\newcommand{\xmark}{\ding{55}}


\begin{document}
\begin{flushright}
    \StrBefore{\currfilename}{.}
\end{flushright}{\it Recall the definitions}: The set of RNA strands $S$ is defined (recursively) by:
\[
\begin{array}{ll}
\textrm{Basis Step: } & \A \in S, \C \in S, \U \in S, \G \in S \\
\textrm{Recursive Step: } & \textrm{If } s \in S\textrm{ and }b \in B \textrm{, then }sb \in S
\end{array}
\]
where $sb$ is string concatenation.
The function \textit{rnalen} that computes the length of RNA strands in $S$ is defined recursively by:
\[
\begin{array}{llll}
& & \textit{rnalen} : S & \to \mathbb{Z}^+ \\
\textrm{Basis Step:} & \textrm{If } b \in B\textrm{ then } & \textit{rnalen}(b) & = 1 \\
\textrm{Recursive Step:} & \textrm{If } s \in S\textrm{ and }b \in B\textrm{, then  } & \textit{rnalen}(sb) & = 1 + \textit{rnalen}(s)
\end{array}
\]
The function \textit{basecount} that computes the number of a given base 
$b$ appearing in a RNA strand $s$ is defined recursively by:
\[
\begin{array}{llll}
& & \textit{basecount} : S \times B & \to \mathbb{N} \\
\textrm{Basis Step:} &  \textrm{If } b_1 \in B, b_2 \in B & \textit{basecount}(~(b_1, b_2)~) & =
        \begin{cases}
            1 & \textrm{when } b_1 = b_2 \\
            0 & \textrm{when } b_1 \neq b_2 \\
        \end{cases} \\
\textrm{Recursive Step:} & \textrm{If } s \in S, b_1 \in B, b_2 \in B &\textit{basecount}(~(s b_1, b_2)~) & =
        \begin{cases}
            1 + \textit{basecount}(~(s, b_2)~) & \textrm{when } b_1 = b_2 \\
            \textit{basecount}(~(s, b_2)~) & \textrm{when } b_1 \neq b_2 \\
        \end{cases}
\end{array}
\]
\end{document}
\input{../activity-snippets/predicates-example-rnalen-basecount.tex}
\input{../activity-snippets/predicates-projecting-example-rna-basecount.tex}
\input{../activity-snippets/predicate-notation.tex}
\newpage
\input{../activity-snippets/nested-quantifiers.tex}
\input{../activity-snippets/alternating-quantifiers-strategies-rna-examples.tex}
\newpage
\subsection*{Review}
\begin{enumerate}
\item \hspace{1in}\\\input{../activity-snippets/quiz-predicates-alternating-quantifiers-rnalen.tex}
\item \hspace{1in}\\\input{../activity-snippets/quiz-predicates-alternating-quantifiers-basecount.tex}
\end{enumerate}


\newpage

\section*{Monday October 25}
\subsection*{Proof strategies}
\input{../activity-snippets/proof-strategies-road-map.tex}
\input{../activity-snippets/proof-strategies-quantification-finite-domain.tex}
\input{../activity-snippets/proof-strategy-universal-exhaustion.tex}
\input{../activity-snippets/proof-strategy-universal-generalization.tex}
\newpage
\documentclass[12pt, oneside]{article}

\usepackage[letterpaper, scale=0.89, centering]{geometry}
\usepackage{fancyhdr}
\setlength{\parindent}{0em}
\setlength{\parskip}{1em}

\pagestyle{fancy}
\fancyhf{}
\renewcommand{\headrulewidth}{0pt}
\rfoot{\href{https://creativecommons.org/licenses/by-nc-sa/2.0/}{CC BY-NC-SA 2.0} Version \today~(\thepage)}

\usepackage{amssymb,amsmath,pifont,amsfonts,comment,enumerate,enumitem}
\usepackage{currfile,xstring,hyperref,tabularx,graphicx,wasysym}
\usepackage[labelformat=empty]{caption}
\usepackage[dvipsnames,table]{xcolor}
\usepackage{multicol,multirow,array,listings,tabularx,lastpage,textcomp,booktabs}

% NOTE(joe): This environment is credit @pnpo (https://tex.stackexchange.com/a/218450)
\lstnewenvironment{algorithm}[1][] %defines the algorithm listing environment
{   
    \lstset{ %this is the stype
        mathescape=true,
        frame=tB,
        numbers=left, 
        numberstyle=\tiny,
        basicstyle=\rmfamily\scriptsize, 
        keywordstyle=\color{black}\bfseries,
        keywords={,procedure, div, for, to, input, output, return, datatype, function, in, if, else, foreach, while, begin, end, }
        numbers=left,
        xleftmargin=.04\textwidth,
        #1
    }
}
{}
\lstnewenvironment{java}[1][]
{   
    \lstset{
        language=java,
        mathescape=true,
        frame=tB,
        numbers=left, 
        numberstyle=\tiny,
        basicstyle=\ttfamily\scriptsize, 
        keywordstyle=\color{black}\bfseries,
        keywords={, int, double, for, return, if, else, while, }
        numbers=left,
        xleftmargin=.04\textwidth,
        #1
    }
}
{}

\newcommand\abs[1]{\lvert~#1~\rvert}
\newcommand{\st}{\mid}

\newcommand{\A}[0]{\texttt{A}}
\newcommand{\C}[0]{\texttt{C}}
\newcommand{\G}[0]{\texttt{G}}
\newcommand{\U}[0]{\texttt{U}}

\newcommand{\cmark}{\ding{51}}
\newcommand{\xmark}{\ding{55}}


\begin{document}
\begin{flushright}
    \StrBefore{\currfilename}{.}
\end{flushright}{\bf Definitions}:
A {\bf set} is an  unordered collection of  elements.
When $A$ and  $B$ are sets,  $A = B$ (set equality) means  
\[
    \forall x  ( x\in A \leftrightarrow x \in B)
\]
When $A$ and  $B$ are sets, $A \subseteq B$ (``$A$ is a {\bf subset} of $B$") means 
\[
    \forall x  (x \in A  \to x  \in B)
\]
When $A$ and  $B$ are sets,  $A \subsetneq B$ (``$A$ is a {\bf proper subset} of $B$") means 
\[
    (A\subseteq B) \wedge  (A \neq B)
\]
\end{document}
\input{../activity-snippets/proof-strategies-conditionals.tex}
\input{../activity-snippets/proof-strategies-proof-by-cases.tex}
\input{../activity-snippets/proof-strategies-ands.tex}
\input{../activity-snippets/sets-proof-strategies.tex}
\input{../activity-snippets/sets-equality-example.tex}
\newpage
\input{../activity-snippets/sets-basic-proofs.tex}
\vfill
\input{../activity-snippets/proofs-signposting.tex}
\newpage
\subsection*{Review}
\begin{enumerate}
\item \hspace{1in}\\ \input{../activity-snippets/quiz-translating-counting-quantifiers.tex}
\item \hspace{1in}\\ \input{../activity-snippets/quiz-sets-claims-subset-equality.tex}
\item We want to hear how the term and this class are going for you.
Please complete the midquarter feedback form: \href{https://forms.gle/w3D7ifAWnD5sWwHf9}{https://forms.gle/w3D7ifAWnD5sWwHf9}
\end{enumerate}

\newpage
\section*{Wednesday October 27}
\input{../activity-snippets/set-operations-union-intersection-powerset.tex}
\newpage
\input{../activity-snippets/sets-basic-proofs-operations.tex}
\newpage
\subsection*{Review}
\begin{enumerate}
\item \hspace{1in}\\ \input{../activity-snippets/quiz-sets-claims.tex}
\item \hspace{1in}\\ \input{../activity-snippets/quiz-sets-proof-strategies.tex}
\end{enumerate}

\newpage
\section*{Friday October 29}
\subsection*{Facts about numbers}
\input{../activity-snippets/numbers-facts.tex}
\subsection*{Factoring}
\documentclass[12pt, oneside]{article}

\usepackage[letterpaper, scale=0.89, centering]{geometry}
\usepackage{fancyhdr}
\setlength{\parindent}{0em}
\setlength{\parskip}{1em}

\pagestyle{fancy}
\fancyhf{}
\renewcommand{\headrulewidth}{0pt}
\rfoot{\href{https://creativecommons.org/licenses/by-nc-sa/2.0/}{CC BY-NC-SA 2.0} Version \today~(\thepage)}

\usepackage{amssymb,amsmath,pifont,amsfonts,comment,enumerate,enumitem}
\usepackage{currfile,xstring,hyperref,tabularx,graphicx,wasysym}
\usepackage[labelformat=empty]{caption}
\usepackage[dvipsnames,table]{xcolor}
\usepackage{multicol,multirow,array,listings,tabularx,lastpage,textcomp,booktabs}

% NOTE(joe): This environment is credit @pnpo (https://tex.stackexchange.com/a/218450)
\lstnewenvironment{algorithm}[1][] %defines the algorithm listing environment
{   
    \lstset{ %this is the stype
        mathescape=true,
        frame=tB,
        numbers=left, 
        numberstyle=\tiny,
        basicstyle=\rmfamily\scriptsize, 
        keywordstyle=\color{black}\bfseries,
        keywords={,procedure, div, for, to, input, output, return, datatype, function, in, if, else, foreach, while, begin, end, }
        numbers=left,
        xleftmargin=.04\textwidth,
        #1
    }
}
{}
\lstnewenvironment{java}[1][]
{   
    \lstset{
        language=java,
        mathescape=true,
        frame=tB,
        numbers=left, 
        numberstyle=\tiny,
        basicstyle=\ttfamily\scriptsize, 
        keywordstyle=\color{black}\bfseries,
        keywords={, int, double, for, return, if, else, while, }
        numbers=left,
        xleftmargin=.04\textwidth,
        #1
    }
}
{}

\newcommand\abs[1]{\lvert~#1~\rvert}
\newcommand{\st}{\mid}

\newcommand{\A}[0]{\texttt{A}}
\newcommand{\C}[0]{\texttt{C}}
\newcommand{\G}[0]{\texttt{G}}
\newcommand{\U}[0]{\texttt{U}}

\newcommand{\cmark}{\ding{51}}
\newcommand{\xmark}{\ding{55}}


\begin{document}
\begin{flushright}
    \StrBefore{\currfilename}{.}
\end{flushright}{\bf Definition}: When $a$ and $b$ are integers and $a$ is nonzero, 
{\bf $a$ divides $b$} means there is an integer $c$ such that $b = ac$ . 
Symbolically, $F(~(a,b)~) = \phantom{\exists c\in \mathbb{Z}~(b=ac)}$
and is  a predicate over the domain \underline{\phantom{$\mathbb{Z}^{\neq 0} \times \mathbb{Z}$}}
Other (synonymous) ways to say that $F(~(a,b)~)$ is true: 
\begin{center}
$a$ is a {\bf factor} of $b$
\qquad 
$a$ is a {\bf divisor} of $b$
\qquad  $b$ is a {\bf multiple} of $a$
\qquad
$a | b$
\end{center}
When $a$ is a positive integer and $b$ is any integer, $a | b$
exactly when $b \textbf{ mod } a = 0$
When $a$ is a positive integer and $b$ is any integer, $a | b$
exactly $b = a \cdot (b \textbf{ div } a)$
\end{document}
\input{../activity-snippets/factoring-translation-examples.tex}
\input{../activity-snippets/factoring-basic-claims.tex}
\newpage
\input{../activity-snippets/factoring-basic-claims-continued.tex}
\input{../activity-snippets/factoring-even-odd.tex}
\documentclass[12pt, oneside]{article}

\usepackage[letterpaper, scale=0.89, centering]{geometry}
\usepackage{fancyhdr}
\setlength{\parindent}{0em}
\setlength{\parskip}{1em}

\pagestyle{fancy}
\fancyhf{}
\renewcommand{\headrulewidth}{0pt}
\rfoot{\href{https://creativecommons.org/licenses/by-nc-sa/2.0/}{CC BY-NC-SA 2.0} Version \today~(\thepage)}

\usepackage{amssymb,amsmath,pifont,amsfonts,comment,enumerate,enumitem}
\usepackage{currfile,xstring,hyperref,tabularx,graphicx,wasysym}
\usepackage[labelformat=empty]{caption}
\usepackage[dvipsnames,table]{xcolor}
\usepackage{multicol,multirow,array,listings,tabularx,lastpage,textcomp,booktabs}

% NOTE(joe): This environment is credit @pnpo (https://tex.stackexchange.com/a/218450)
\lstnewenvironment{algorithm}[1][] %defines the algorithm listing environment
{   
    \lstset{ %this is the stype
        mathescape=true,
        frame=tB,
        numbers=left, 
        numberstyle=\tiny,
        basicstyle=\rmfamily\scriptsize, 
        keywordstyle=\color{black}\bfseries,
        keywords={,procedure, div, for, to, input, output, return, datatype, function, in, if, else, foreach, while, begin, end, }
        numbers=left,
        xleftmargin=.04\textwidth,
        #1
    }
}
{}
\lstnewenvironment{java}[1][]
{   
    \lstset{
        language=java,
        mathescape=true,
        frame=tB,
        numbers=left, 
        numberstyle=\tiny,
        basicstyle=\ttfamily\scriptsize, 
        keywordstyle=\color{black}\bfseries,
        keywords={, int, double, for, return, if, else, while, }
        numbers=left,
        xleftmargin=.04\textwidth,
        #1
    }
}
{}

\newcommand\abs[1]{\lvert~#1~\rvert}
\newcommand{\st}{\mid}

\newcommand{\A}[0]{\texttt{A}}
\newcommand{\C}[0]{\texttt{C}}
\newcommand{\G}[0]{\texttt{G}}
\newcommand{\U}[0]{\texttt{U}}

\newcommand{\cmark}{\ding{51}}
\newcommand{\xmark}{\ding{55}}


\begin{document}
\begin{flushright}
    \StrBefore{\currfilename}{.}
\end{flushright}{\bf Definition}:  An integer $p$ greater than $1$ is called {\bf prime} means 
the only positive factors of 
$p$ are $1$ and $p$. A positive integer that is greater than $1$ and is not prime 
is called composite.
\end{document}
\input{../activity-snippets/primes-basic-claims.tex}
\newpage
\subsection*{Review}
\begin{enumerate}
    \item \hspace{1in}\\ \input{../activity-snippets/quiz-factoring-quantifiers.tex}
    \item \hspace{1in}\\ \documentclass[12pt, oneside]{article}

\usepackage[letterpaper, scale=0.89, centering]{geometry}
\usepackage{fancyhdr}
\setlength{\parindent}{0em}
\setlength{\parskip}{1em}

\pagestyle{fancy}
\fancyhf{}
\renewcommand{\headrulewidth}{0pt}
\rfoot{\href{https://creativecommons.org/licenses/by-nc-sa/2.0/}{CC BY-NC-SA 2.0} Version \today~(\thepage)}

\usepackage{amssymb,amsmath,pifont,amsfonts,comment,enumerate,enumitem}
\usepackage{currfile,xstring,hyperref,tabularx,graphicx,wasysym}
\usepackage[labelformat=empty]{caption}
\usepackage[dvipsnames,table]{xcolor}
\usepackage{multicol,multirow,array,listings,tabularx,lastpage,textcomp,booktabs}

% NOTE(joe): This environment is credit @pnpo (https://tex.stackexchange.com/a/218450)
\lstnewenvironment{algorithm}[1][] %defines the algorithm listing environment
{   
    \lstset{ %this is the stype
        mathescape=true,
        frame=tB,
        numbers=left, 
        numberstyle=\tiny,
        basicstyle=\rmfamily\scriptsize, 
        keywordstyle=\color{black}\bfseries,
        keywords={,procedure, div, for, to, input, output, return, datatype, function, in, if, else, foreach, while, begin, end, }
        numbers=left,
        xleftmargin=.04\textwidth,
        #1
    }
}
{}
\lstnewenvironment{java}[1][]
{   
    \lstset{
        language=java,
        mathescape=true,
        frame=tB,
        numbers=left, 
        numberstyle=\tiny,
        basicstyle=\ttfamily\scriptsize, 
        keywordstyle=\color{black}\bfseries,
        keywords={, int, double, for, return, if, else, while, }
        numbers=left,
        xleftmargin=.04\textwidth,
        #1
    }
}
{}

\newcommand\abs[1]{\lvert~#1~\rvert}
\newcommand{\st}{\mid}

\newcommand{\A}[0]{\texttt{A}}
\newcommand{\C}[0]{\texttt{C}}
\newcommand{\G}[0]{\texttt{G}}
\newcommand{\U}[0]{\texttt{U}}

\newcommand{\cmark}{\ding{51}}
\newcommand{\xmark}{\ding{55}}


\begin{document}
\begin{flushright}
    \StrBefore{\currfilename}{.}
\end{flushright}Which of the following formalizes the definition of the predicate
$Pr(x)$ over the set of integers, and evaluates to $T$ exactly when 
$x$ is prime. (Select all and only correct options.)
\begin{enumerate}
    \item $\forall a \in \mathbb{Z}^{\neq 0}~( ~(x > 1 \land a >0) \to F(~(a,x)~))$
    \item $\lnot \exists a \in \mathbb{Z}^{\neq 0} ~(x > 1 \land (a=1 \lor a=x) \land F(~(a,x)~))$
    \item $(x > 1) \land \forall a \in \mathbb{Z}^{\neq 0}~( ~(~ a>0 \land F(~(a,x)~)~) \to (a=1 \lor a=x)~)$
    \item $(x > 1) \land \forall a \in \mathbb{Z}^{\neq 0}~( ~(~ a>1 \land \lnot (a=x) ~) \to \lnot F(~(a,x)~)~)$
\end{enumerate}
\end{document}
\end{enumerate}


\newpage

\section*{Monday November 1}

Today's session is on Zoom, log in with your @ucsd.edu account https://ucsd.zoom.us/j/97431852722
Meeting ID: 974 3185 2722

\documentclass[12pt, oneside]{article}

\usepackage[letterpaper, scale=0.89, centering]{geometry}
\usepackage{fancyhdr}
\setlength{\parindent}{0em}
\setlength{\parskip}{1em}

\pagestyle{fancy}
\fancyhf{}
\renewcommand{\headrulewidth}{0pt}
\rfoot{\href{https://creativecommons.org/licenses/by-nc-sa/2.0/}{CC BY-NC-SA 2.0} Version \today~(\thepage)}

\usepackage{amssymb,amsmath,pifont,amsfonts,comment,enumerate,enumitem}
\usepackage{currfile,xstring,hyperref,tabularx,graphicx,wasysym}
\usepackage[labelformat=empty]{caption}
\usepackage[dvipsnames,table]{xcolor}
\usepackage{multicol,multirow,array,listings,tabularx,lastpage,textcomp,booktabs}

% NOTE(joe): This environment is credit @pnpo (https://tex.stackexchange.com/a/218450)
\lstnewenvironment{algorithm}[1][] %defines the algorithm listing environment
{   
    \lstset{ %this is the stype
        mathescape=true,
        frame=tB,
        numbers=left, 
        numberstyle=\tiny,
        basicstyle=\rmfamily\scriptsize, 
        keywordstyle=\color{black}\bfseries,
        keywords={,procedure, div, for, to, input, output, return, datatype, function, in, if, else, foreach, while, begin, end, }
        numbers=left,
        xleftmargin=.04\textwidth,
        #1
    }
}
{}
\lstnewenvironment{java}[1][]
{   
    \lstset{
        language=java,
        mathescape=true,
        frame=tB,
        numbers=left, 
        numberstyle=\tiny,
        basicstyle=\ttfamily\scriptsize, 
        keywordstyle=\color{black}\bfseries,
        keywords={, int, double, for, return, if, else, while, }
        numbers=left,
        xleftmargin=.04\textwidth,
        #1
    }
}
{}

\newcommand\abs[1]{\lvert~#1~\rvert}
\newcommand{\st}{\mid}

\newcommand{\A}[0]{\texttt{A}}
\newcommand{\C}[0]{\texttt{C}}
\newcommand{\G}[0]{\texttt{G}}
\newcommand{\U}[0]{\texttt{U}}

\newcommand{\cmark}{\ding{51}}
\newcommand{\xmark}{\ding{55}}


\begin{document}
\begin{flushright}
    \StrBefore{\currfilename}{.}
\end{flushright}{\it Recall the definitions}: The set of RNA strands $S$ is defined (recursively) by:
\[
\begin{array}{ll}
\textrm{Basis Step: } & \A \in S, \C \in S, \U \in S, \G \in S \\
\textrm{Recursive Step: } & \textrm{If } s \in S\textrm{ and }b \in B \textrm{, then }sb \in S
\end{array}
\]
where $sb$ is string concatenation.
The function \textit{rnalen} that computes the length of RNA strands in $S$ is defined recursively by:
\[
\begin{array}{llll}
& & \textit{rnalen} : S & \to \mathbb{Z}^+ \\
\textrm{Basis Step:} & \textrm{If } b \in B\textrm{ then } & \textit{rnalen}(b) & = 1 \\
\textrm{Recursive Step:} & \textrm{If } s \in S\textrm{ and }b \in B\textrm{, then  } & \textit{rnalen}(sb) & = 1 + \textit{rnalen}(s)
\end{array}
\]
The function \textit{basecount} that computes the number of a given base 
$b$ appearing in a RNA strand $s$ is defined recursively by:
\[
\begin{array}{llll}
& & \textit{basecount} : S \times B & \to \mathbb{N} \\
\textrm{Basis Step:} &  \textrm{If } b_1 \in B, b_2 \in B & \textit{basecount}(~(b_1, b_2)~) & =
        \begin{cases}
            1 & \textrm{when } b_1 = b_2 \\
            0 & \textrm{when } b_1 \neq b_2 \\
        \end{cases} \\
\textrm{Recursive Step:} & \textrm{If } s \in S, b_1 \in B, b_2 \in B &\textit{basecount}(~(s b_1, b_2)~) & =
        \begin{cases}
            1 + \textit{basecount}(~(s, b_2)~) & \textrm{when } b_1 = b_2 \\
            \textit{basecount}(~(s, b_2)~) & \textrm{when } b_1 \neq b_2 \\
        \end{cases}
\end{array}
\]
\end{document}

At this point, we've seen the proof strategies
\begin{multicols}{2}
    \begin{itemize}
        \item A {\bf counterexample} to prove that  $\forall x P(x)$ is {\bf false}.
        \item  A {\bf witness} to prove that  $\exists x P(x)$ is {\bf true}.
        \item {\bf Proof of universal by exhaustion} to prove that $\forall x \, P(x)$
    is true when $P$ has a finite domain
        \item  {\bf Proof by universal generalization} to prove that $\forall x \, P(x)$
    is true using an arbitrary element of the domain.
        \item To  prove  that $\exists x P(x)$ is {\bf false}, write the universal statement that is 
        logically equivalent to its negation and then prove it true using universal generalization.
        \item To prove that $p \land q$ is true, have two subgoals: 
        subgoal (1) prove $p$ is  true; and, subgoal (2) prove $q$ is true. To prove that $p \land q$ is false, it's enough to prove that $p$ is false.
     To prove that $p \land q$ is false, it's enough to prove that $q$ is false.
        \item Proof of conditional by {\bf direct proof}
        \item Proof of conditional by {\bf contrapositive proof}
        \item Proof of disjuction using equivalent conditional: To prove that the 
        disjunction $p \lor q$ is true, we can rewrite it equivalently as $\lnot p \to q$ and
        then use direct proof or contrapositive proof.
        \item {\bf Proof by cases}.
    \end{itemize}
\end{multicols}
\newpage
\input{../activity-snippets/alternating-quantifiers-proofs-rna-examples.tex}
\newpage
\input{../activity-snippets/structural-induction-motivating-example-rna.tex}
\input{../activity-snippets/proof-strategies-structural-induction.tex}
\newpage
\input{../activity-snippets/structural-induction-example-rnalen-basecount.tex}
\newpage
\subsection*{Review}
\input{../activity-snippets/quiz-basecount-rnalen-induction.tex}

\newpage
\section*{Wednesday November 3}
\input{../activity-snippets/proofs-signposting-kinds-of-claims.tex}
\input{../activity-snippets/structural-induction-example-robot-grid.tex}
\newpage
\input{../activity-snippets/structural-induction-example-sum-of-powers.tex}
\vfill
\input{../activity-snippets/proof-strategy-mathematical-induction.tex}
\newpage
\subsection*{Review}
\begin{enumerate}
\item \hspace{1in}\\ \input{../activity-snippets/quiz-robot-grid.tex}
\item \hspace{1in}\\ \input{../activity-snippets/quiz-comparing-structural-mathematical-induction.tex}
\newpage
\item \hspace{1in}\\ \input{../activity-snippets/quiz-exponential-factorial.tex}
\end{enumerate}

\newpage
\section*{Friday November 5}
\documentclass[12pt, oneside]{article}

\usepackage[letterpaper, scale=0.89, centering]{geometry}
\usepackage{fancyhdr}
\setlength{\parindent}{0em}
\setlength{\parskip}{1em}

\pagestyle{fancy}
\fancyhf{}
\renewcommand{\headrulewidth}{0pt}
\rfoot{\href{https://creativecommons.org/licenses/by-nc-sa/2.0/}{CC BY-NC-SA 2.0} Version \today~(\thepage)}

\usepackage{amssymb,amsmath,pifont,amsfonts,comment,enumerate,enumitem}
\usepackage{currfile,xstring,hyperref,tabularx,graphicx,wasysym}
\usepackage[labelformat=empty]{caption}
\usepackage[dvipsnames,table]{xcolor}
\usepackage{multicol,multirow,array,listings,tabularx,lastpage,textcomp,booktabs}

% NOTE(joe): This environment is credit @pnpo (https://tex.stackexchange.com/a/218450)
\lstnewenvironment{algorithm}[1][] %defines the algorithm listing environment
{   
    \lstset{ %this is the stype
        mathescape=true,
        frame=tB,
        numbers=left, 
        numberstyle=\tiny,
        basicstyle=\rmfamily\scriptsize, 
        keywordstyle=\color{black}\bfseries,
        keywords={,procedure, div, for, to, input, output, return, datatype, function, in, if, else, foreach, while, begin, end, }
        numbers=left,
        xleftmargin=.04\textwidth,
        #1
    }
}
{}
\lstnewenvironment{java}[1][]
{   
    \lstset{
        language=java,
        mathescape=true,
        frame=tB,
        numbers=left, 
        numberstyle=\tiny,
        basicstyle=\ttfamily\scriptsize, 
        keywordstyle=\color{black}\bfseries,
        keywords={, int, double, for, return, if, else, while, }
        numbers=left,
        xleftmargin=.04\textwidth,
        #1
    }
}
{}

\newcommand\abs[1]{\lvert~#1~\rvert}
\newcommand{\st}{\mid}

\newcommand{\A}[0]{\texttt{A}}
\newcommand{\C}[0]{\texttt{C}}
\newcommand{\G}[0]{\texttt{G}}
\newcommand{\U}[0]{\texttt{U}}

\newcommand{\cmark}{\ding{51}}
\newcommand{\xmark}{\ding{55}}


\begin{document}
\begin{flushright}
    \StrBefore{\currfilename}{.}
\end{flushright}{\bf Definition} The set of linked lists of natural numbers $L$ is defined recursively by
\[
\begin{array}{ll}
    \textrm{Basis Step: } & [] \in L \\
    \textrm{Recursive Step: } & \textrm{If } l \in L\textrm{ and }n \in \mathbb{N} \textrm{, then } (n, l) \in L
\end{array}
\]
\end{document}
\input{../activity-snippets/linked-lists-examples.tex}
\documentclass[12pt, oneside]{article}

\usepackage[letterpaper, scale=0.89, centering]{geometry}
\usepackage{fancyhdr}
\setlength{\parindent}{0em}
\setlength{\parskip}{1em}

\pagestyle{fancy}
\fancyhf{}
\renewcommand{\headrulewidth}{0pt}
\rfoot{\href{https://creativecommons.org/licenses/by-nc-sa/2.0/}{CC BY-NC-SA 2.0} Version \today~(\thepage)}

\usepackage{amssymb,amsmath,pifont,amsfonts,comment,enumerate,enumitem}
\usepackage{currfile,xstring,hyperref,tabularx,graphicx,wasysym}
\usepackage[labelformat=empty]{caption}
\usepackage[dvipsnames,table]{xcolor}
\usepackage{multicol,multirow,array,listings,tabularx,lastpage,textcomp,booktabs}

% NOTE(joe): This environment is credit @pnpo (https://tex.stackexchange.com/a/218450)
\lstnewenvironment{algorithm}[1][] %defines the algorithm listing environment
{   
    \lstset{ %this is the stype
        mathescape=true,
        frame=tB,
        numbers=left, 
        numberstyle=\tiny,
        basicstyle=\rmfamily\scriptsize, 
        keywordstyle=\color{black}\bfseries,
        keywords={,procedure, div, for, to, input, output, return, datatype, function, in, if, else, foreach, while, begin, end, }
        numbers=left,
        xleftmargin=.04\textwidth,
        #1
    }
}
{}
\lstnewenvironment{java}[1][]
{   
    \lstset{
        language=java,
        mathescape=true,
        frame=tB,
        numbers=left, 
        numberstyle=\tiny,
        basicstyle=\ttfamily\scriptsize, 
        keywordstyle=\color{black}\bfseries,
        keywords={, int, double, for, return, if, else, while, }
        numbers=left,
        xleftmargin=.04\textwidth,
        #1
    }
}
{}

\newcommand\abs[1]{\lvert~#1~\rvert}
\newcommand{\st}{\mid}

\newcommand{\A}[0]{\texttt{A}}
\newcommand{\C}[0]{\texttt{C}}
\newcommand{\G}[0]{\texttt{G}}
\newcommand{\U}[0]{\texttt{U}}

\newcommand{\cmark}{\ding{51}}
\newcommand{\xmark}{\ding{55}}


\begin{document}
\begin{flushright}
    \StrBefore{\currfilename}{.}
\end{flushright}{\bf Definition}: The length of a linked list of natural numbers $L$, $length: L \to \mathbb{N}$ is defined by
\[
\begin{array}{llll}
\textrm{Basis Step:} &  & length(~[]~) &= 0 \\
\textrm{Recursive Step:} & \textrm{If } l \in L\textrm{ and }n \in \mathbb{N}\textrm{, then  } & length(~(n, l)~)  &= 1+ length(l)
\end{array}
\]

\end{document}
\vspace{50pt}
\documentclass[12pt, oneside]{article}

\usepackage[letterpaper, scale=0.89, centering]{geometry}
\usepackage{fancyhdr}
\setlength{\parindent}{0em}
\setlength{\parskip}{1em}

\pagestyle{fancy}
\fancyhf{}
\renewcommand{\headrulewidth}{0pt}
\rfoot{\href{https://creativecommons.org/licenses/by-nc-sa/2.0/}{CC BY-NC-SA 2.0} Version \today~(\thepage)}

\usepackage{amssymb,amsmath,pifont,amsfonts,comment,enumerate,enumitem}
\usepackage{currfile,xstring,hyperref,tabularx,graphicx,wasysym}
\usepackage[labelformat=empty]{caption}
\usepackage[dvipsnames,table]{xcolor}
\usepackage{multicol,multirow,array,listings,tabularx,lastpage,textcomp,booktabs}

% NOTE(joe): This environment is credit @pnpo (https://tex.stackexchange.com/a/218450)
\lstnewenvironment{algorithm}[1][] %defines the algorithm listing environment
{   
    \lstset{ %this is the stype
        mathescape=true,
        frame=tB,
        numbers=left, 
        numberstyle=\tiny,
        basicstyle=\rmfamily\scriptsize, 
        keywordstyle=\color{black}\bfseries,
        keywords={,procedure, div, for, to, input, output, return, datatype, function, in, if, else, foreach, while, begin, end, }
        numbers=left,
        xleftmargin=.04\textwidth,
        #1
    }
}
{}
\lstnewenvironment{java}[1][]
{   
    \lstset{
        language=java,
        mathescape=true,
        frame=tB,
        numbers=left, 
        numberstyle=\tiny,
        basicstyle=\ttfamily\scriptsize, 
        keywordstyle=\color{black}\bfseries,
        keywords={, int, double, for, return, if, else, while, }
        numbers=left,
        xleftmargin=.04\textwidth,
        #1
    }
}
{}

\newcommand\abs[1]{\lvert~#1~\rvert}
\newcommand{\st}{\mid}

\newcommand{\A}[0]{\texttt{A}}
\newcommand{\C}[0]{\texttt{C}}
\newcommand{\G}[0]{\texttt{G}}
\newcommand{\U}[0]{\texttt{U}}

\newcommand{\cmark}{\ding{51}}
\newcommand{\xmark}{\ding{55}}


\begin{document}
\begin{flushright}
    \StrBefore{\currfilename}{.}
\end{flushright}{\bf Definition}: The function $prepend : L \times \mathbb{N} \to L$ that adds an element at the 
front of a linked list is defined by
\[
\phantom{prepend(~(l, n)~) = (n, l)}
\]

\end{document}
\vspace{50pt}
\documentclass[12pt, oneside]{article}

\usepackage[letterpaper, scale=0.89, centering]{geometry}
\usepackage{fancyhdr}
\setlength{\parindent}{0em}
\setlength{\parskip}{1em}

\pagestyle{fancy}
\fancyhf{}
\renewcommand{\headrulewidth}{0pt}
\rfoot{\href{https://creativecommons.org/licenses/by-nc-sa/2.0/}{CC BY-NC-SA 2.0} Version \today~(\thepage)}

\usepackage{amssymb,amsmath,pifont,amsfonts,comment,enumerate,enumitem}
\usepackage{currfile,xstring,hyperref,tabularx,graphicx,wasysym}
\usepackage[labelformat=empty]{caption}
\usepackage[dvipsnames,table]{xcolor}
\usepackage{multicol,multirow,array,listings,tabularx,lastpage,textcomp,booktabs}

% NOTE(joe): This environment is credit @pnpo (https://tex.stackexchange.com/a/218450)
\lstnewenvironment{algorithm}[1][] %defines the algorithm listing environment
{   
    \lstset{ %this is the stype
        mathescape=true,
        frame=tB,
        numbers=left, 
        numberstyle=\tiny,
        basicstyle=\rmfamily\scriptsize, 
        keywordstyle=\color{black}\bfseries,
        keywords={,procedure, div, for, to, input, output, return, datatype, function, in, if, else, foreach, while, begin, end, }
        numbers=left,
        xleftmargin=.04\textwidth,
        #1
    }
}
{}
\lstnewenvironment{java}[1][]
{   
    \lstset{
        language=java,
        mathescape=true,
        frame=tB,
        numbers=left, 
        numberstyle=\tiny,
        basicstyle=\ttfamily\scriptsize, 
        keywordstyle=\color{black}\bfseries,
        keywords={, int, double, for, return, if, else, while, }
        numbers=left,
        xleftmargin=.04\textwidth,
        #1
    }
}
{}

\newcommand\abs[1]{\lvert~#1~\rvert}
\newcommand{\st}{\mid}

\newcommand{\A}[0]{\texttt{A}}
\newcommand{\C}[0]{\texttt{C}}
\newcommand{\G}[0]{\texttt{G}}
\newcommand{\U}[0]{\texttt{U}}

\newcommand{\cmark}{\ding{51}}
\newcommand{\xmark}{\ding{55}}


\begin{document}
\begin{flushright}
    \StrBefore{\currfilename}{.}
\end{flushright}{\bf Definition} The function $append : L \times \mathbb{N} \to L$ that 
adds an element at the end of a linked list is defined by
\[
\begin{array}{llll}
\textrm{Basis Step:} & \textrm{If } m \in \mathbb{N}\textrm{ then } & \phantom{append(~([], m)~)} & \phantom{= (m, []) }\\
\textrm{Recursive Step:} & \textrm{If } l \in L\textrm{ and }n \in \mathbb{N}\textrm{ and }m \in \mathbb{N}\textrm{, then  } & \phantom{append(~(~(n, l), m~)~) } &\phantom{= (n, append(~(l, m)~)~)}
\end{array}
\]
\end{document}
\vspace{50pt}
\newpage
\input{../activity-snippets/linked-list-append-length-claim-proof.tex}
\newpage
\input{../activity-snippets/linked-list-example-each-length.tex}
\newpage
\subsection*{Review}
\documentclass[12pt, oneside]{article}

\usepackage[letterpaper, scale=0.89, centering]{geometry}
\usepackage{fancyhdr}
\setlength{\parindent}{0em}
\setlength{\parskip}{1em}

\pagestyle{fancy}
\fancyhf{}
\renewcommand{\headrulewidth}{0pt}
\rfoot{\href{https://creativecommons.org/licenses/by-nc-sa/2.0/}{CC BY-NC-SA 2.0} Version \today~(\thepage)}

\usepackage{amssymb,amsmath,pifont,amsfonts,comment,enumerate,enumitem}
\usepackage{currfile,xstring,hyperref,tabularx,graphicx,wasysym}
\usepackage[labelformat=empty]{caption}
\usepackage[dvipsnames,table]{xcolor}
\usepackage{multicol,multirow,array,listings,tabularx,lastpage,textcomp,booktabs}

% NOTE(joe): This environment is credit @pnpo (https://tex.stackexchange.com/a/218450)
\lstnewenvironment{algorithm}[1][] %defines the algorithm listing environment
{   
    \lstset{ %this is the stype
        mathescape=true,
        frame=tB,
        numbers=left, 
        numberstyle=\tiny,
        basicstyle=\rmfamily\scriptsize, 
        keywordstyle=\color{black}\bfseries,
        keywords={,procedure, div, for, to, input, output, return, datatype, function, in, if, else, foreach, while, begin, end, }
        numbers=left,
        xleftmargin=.04\textwidth,
        #1
    }
}
{}
\lstnewenvironment{java}[1][]
{   
    \lstset{
        language=java,
        mathescape=true,
        frame=tB,
        numbers=left, 
        numberstyle=\tiny,
        basicstyle=\ttfamily\scriptsize, 
        keywordstyle=\color{black}\bfseries,
        keywords={, int, double, for, return, if, else, while, }
        numbers=left,
        xleftmargin=.04\textwidth,
        #1
    }
}
{}

\newcommand\abs[1]{\lvert~#1~\rvert}
\newcommand{\st}{\mid}

\newcommand{\A}[0]{\texttt{A}}
\newcommand{\C}[0]{\texttt{C}}
\newcommand{\G}[0]{\texttt{G}}
\newcommand{\U}[0]{\texttt{U}}

\newcommand{\cmark}{\ding{51}}
\newcommand{\xmark}{\ding{55}}


\begin{document}
\begin{flushright}
    \StrBefore{\currfilename}{.}
\end{flushright}Recall the definition of linked lists from class.
Consider this (incomplete) definition:
{\bf Definition} The function $\textit{increment} : \underline{\hspace{6em}}$ 
that adds 1 to the data in each node of a linked list is defined by:
\[
\begin{array}{llll}
& & \textit{increment} : \underline{\hspace{3em}} & \to \underline{\hspace{3em}} \\
\textrm{Basis Step:} & & \textit{increment}([]) & = [] \\
\textrm{Recursive Step:} & \textrm{If } l \in L, n \in \mathbb{N} & \textit{increment}((n, l)) & = (1 + n, \textit{increment}(l))
\end{array}
\]
Consider this (incomplete) definition:
{\bf Definition} The function $\textit{sum} : L \to \mathbb{N}$ that adds 
together all the data in nodes of the list is defined by:
\[
\begin{array}{llll}
& & \textit{sum} : L & \to \mathbb{N} \\
\textrm{Basis Step:} & & \textit{sum}([]) & = 0 \\
\textrm{Recursive Step:} & \textrm{If } l \in L, n \in \mathbb{N} & \textit{sum}((n, l)) & = \underline{\hspace{8em}}
\end{array}
\]
You will compute a sample function application and then fill in the 
blanks for the domain and codomain of each of these functions.
\begin{enumerate}
    \item Based on the definition, what is the result of $\textit{increment}((4, (2, (7, []))))$? Write your answer directly with no spaces.
    
    \item Which of the following describes the domain and codomain of \textit{increment}?
    
    \begin{multicols}{2}
    \begin{enumerate}
        \item The domain is $L$ and the codomain is $\mathbb{N}$
        \item The domain is $L$ and the codomain is $\mathbb{N} \times L$
        \item The domain is $L \times \mathbb{N}$ and the codomain is $L$
        \item The domain is $L \times \mathbb{N}$ and the codomain is $\mathbb{N}$
        \item The domain is $L$ and the codomain is $L$
        \item None of the above
    \end{enumerate}
    \end{multicols}
    
    \item Assuming we would like $sum((5, (6, [])))$ to evaluate to $11$ and $sum((3, (1, (8, []))))$ to evaluate to $12$, which of the following could be used to fill in the definition of the recursive case of \textit{sum}?
    
     \begin{multicols}{2}
    \begin{enumerate}
        \item $\begin{cases}
            1 + \textit{sum}(l) & \textrm{when } n \neq 0 \\
            \textit{sum}(l) & \textrm{when } n = 0 \\
        \end{cases}$
        \item $1 + \textit{sum}(l)$
        \item $n + \textit{increment}(l)$
        \item $n + \textit{sum}(l)$
        \item None of the above
    \end{enumerate}
    \end{multicols}
    
    \newpage
    \item Choose only and all of the following statements that are \textbf{well-defined}; that is, they correctly reflect the domains and codomains of the functions and quantifiers, and respect the notational conventions we use in this class. Note that a well-defined statement may be true or false.
    \begin{multicols}{2}    
    \begin{enumerate}
        \item $\forall l \in L \, (\textit{sum}(l))$
        \item $\exists l \in L \, (\textit{sum}(l) \land \textit{length}(l))$
        \item $\forall l \in L \, (\textit{sum}(\textit{increment}(l)) = 10)$
        \item $\exists l \in L \, (\textit{sum}(\textit{increment}(l)) = 10)$
        \item $\forall l \in L \, \forall n \in \mathbb{N} \, ((n \times l) \subseteq L)$
        \item $\forall l_1 \in L \, \exists l_2 \in L \, (\textit{increment}(\textit{sum}(l_1)) = l_2)$
        \item $\forall l \in L \, (\textit{length}(\textit{increment}(l)) = \textit{length}(l))$
    \end{enumerate}
    \end{multicols}
    
    \item Choose only and all of the statements in the previous part that are both well-defined and true.
\end{enumerate}
\end{document}


\newpage

\section*{Monday November 8}
\subsection*{Visualizing induction}
\input{../activity-snippets/induction-dominos.tex}
\input{../activity-snippets/proof-strategy-mathematical-induction.tex}
\input{../activity-snippets/proof-strategy-strong-induction.tex}
\newpage
\input{../activity-snippets/binary-expansions-exist-proof.tex}

\subsubsection*{Representing positive integers with primes}
\input{../activity-snippets/fundamental-theorem-proof.tex}
\subsubsection*{Sending old-fashioned mail with postage stamps}
\input{../activity-snippets/strong-induction-making-change-proof-idea.tex}
\newpage
\subsection*{Review}
\begin{enumerate}
    \item \hspace{1in} \\ \input{../activity-snippets/quiz-binary-expansions-exist-invalid-proof.tex}
    \item \hspace{1in}\\ \input{../activity-snippets/quiz-prime-factorization.tex}
    \newpage
    \item \input{../activity-snippets/quiz-making-change-proof-two-ways.tex}
\end{enumerate}

\newpage
\section*{Wednesday November 10}

\subsubsection*{Finding a winning strategy for a game}
\input{../activity-snippets/strong-induction-nim.tex}
\newpage
\input{../activity-snippets/proof-strategy-proof-by-contradiction.tex}
\subsection*{Least and greatest}
\input{../activity-snippets/least-greatest-proofs.tex}

\newpage
\subsection*{Review}
\begin{enumerate}
\item \hspace{1in}\\ \input{../activity-snippets/quiz-nim.tex}
\item \hspace{1in}\\ \input{../activity-snippets/quiz-no-greatest-prime.tex}
\item \hspace{1in}\\ \input{../activity-snippets/quiz-choosing-proof-strategy.tex}
\end{enumerate}

\newpage
\section*{Friday November 12}

\documentclass[12pt, oneside]{article}

\usepackage[letterpaper, scale=0.89, centering]{geometry}
\usepackage{fancyhdr}
\setlength{\parindent}{0em}
\setlength{\parskip}{1em}

\pagestyle{fancy}
\fancyhf{}
\renewcommand{\headrulewidth}{0pt}
\rfoot{\href{https://creativecommons.org/licenses/by-nc-sa/2.0/}{CC BY-NC-SA 2.0} Version \today~(\thepage)}

\usepackage{amssymb,amsmath,pifont,amsfonts,comment,enumerate,enumitem}
\usepackage{currfile,xstring,hyperref,tabularx,graphicx,wasysym}
\usepackage[labelformat=empty]{caption}
\usepackage[dvipsnames,table]{xcolor}
\usepackage{multicol,multirow,array,listings,tabularx,lastpage,textcomp,booktabs}

% NOTE(joe): This environment is credit @pnpo (https://tex.stackexchange.com/a/218450)
\lstnewenvironment{algorithm}[1][] %defines the algorithm listing environment
{   
    \lstset{ %this is the stype
        mathescape=true,
        frame=tB,
        numbers=left, 
        numberstyle=\tiny,
        basicstyle=\rmfamily\scriptsize, 
        keywordstyle=\color{black}\bfseries,
        keywords={,procedure, div, for, to, input, output, return, datatype, function, in, if, else, foreach, while, begin, end, }
        numbers=left,
        xleftmargin=.04\textwidth,
        #1
    }
}
{}
\lstnewenvironment{java}[1][]
{   
    \lstset{
        language=java,
        mathescape=true,
        frame=tB,
        numbers=left, 
        numberstyle=\tiny,
        basicstyle=\ttfamily\scriptsize, 
        keywordstyle=\color{black}\bfseries,
        keywords={, int, double, for, return, if, else, while, }
        numbers=left,
        xleftmargin=.04\textwidth,
        #1
    }
}
{}

\newcommand\abs[1]{\lvert~#1~\rvert}
\newcommand{\st}{\mid}

\newcommand{\A}[0]{\texttt{A}}
\newcommand{\C}[0]{\texttt{C}}
\newcommand{\G}[0]{\texttt{G}}
\newcommand{\U}[0]{\texttt{U}}

\newcommand{\cmark}{\ding{51}}
\newcommand{\xmark}{\ding{55}}


\begin{document}
\begin{flushright}
    \StrBefore{\currfilename}{.}
\end{flushright}{\bf Definition}: {\bf Greatest common divisor} Let $a$ and $b$ be integers, not both zero. The largest integer $d$ such that 
$d$ is a  factor of $a$ and $d$ is a factor of  $b$ is called the greatest common divisor of $a$ and $b$ 
and is denoted by $gcd(~(a, b)~)$.
\end{document}
\input{../activity-snippets/gcd-examples.tex}
\input{../activity-snippets/gcd-basic-claims.tex}
\input{../activity-snippets/gcd-lemma-relatively-prime.tex}

\newpage
\subsection*{Sets of numbers}

We've seen multiple representations of the set of positive integers
(using base expansions and using prime factorization). Now we're 
going to expand our attention to other sets of numbers as well.
\documentclass[12pt, oneside]{article}

\usepackage[letterpaper, scale=0.89, centering]{geometry}
\usepackage{fancyhdr}
\setlength{\parindent}{0em}
\setlength{\parskip}{1em}

\pagestyle{fancy}
\fancyhf{}
\renewcommand{\headrulewidth}{0pt}
\rfoot{\href{https://creativecommons.org/licenses/by-nc-sa/2.0/}{CC BY-NC-SA 2.0} Version \today~(\thepage)}

\usepackage{amssymb,amsmath,pifont,amsfonts,comment,enumerate,enumitem}
\usepackage{currfile,xstring,hyperref,tabularx,graphicx,wasysym}
\usepackage[labelformat=empty]{caption}
\usepackage[dvipsnames,table]{xcolor}
\usepackage{multicol,multirow,array,listings,tabularx,lastpage,textcomp,booktabs}

% NOTE(joe): This environment is credit @pnpo (https://tex.stackexchange.com/a/218450)
\lstnewenvironment{algorithm}[1][] %defines the algorithm listing environment
{   
    \lstset{ %this is the stype
        mathescape=true,
        frame=tB,
        numbers=left, 
        numberstyle=\tiny,
        basicstyle=\rmfamily\scriptsize, 
        keywordstyle=\color{black}\bfseries,
        keywords={,procedure, div, for, to, input, output, return, datatype, function, in, if, else, foreach, while, begin, end, }
        numbers=left,
        xleftmargin=.04\textwidth,
        #1
    }
}
{}
\lstnewenvironment{java}[1][]
{   
    \lstset{
        language=java,
        mathescape=true,
        frame=tB,
        numbers=left, 
        numberstyle=\tiny,
        basicstyle=\ttfamily\scriptsize, 
        keywordstyle=\color{black}\bfseries,
        keywords={, int, double, for, return, if, else, while, }
        numbers=left,
        xleftmargin=.04\textwidth,
        #1
    }
}
{}

\newcommand\abs[1]{\lvert~#1~\rvert}
\newcommand{\st}{\mid}

\newcommand{\A}[0]{\texttt{A}}
\newcommand{\C}[0]{\texttt{C}}
\newcommand{\G}[0]{\texttt{G}}
\newcommand{\U}[0]{\texttt{U}}

\newcommand{\cmark}{\ding{51}}
\newcommand{\xmark}{\ding{55}}


\begin{document}
\begin{flushright}
    \StrBefore{\currfilename}{.}
\end{flushright}The {\bf set  of rational numbers}, $\mathbb{Q}$  is defined as 
\[
\left\{ \frac{p}{q} \mid p \in \mathbb{Z}  \text{ and  } q  \in \mathbb{Z} \text{ and } q \neq  0 \right\}
\text{~~~~or, equivalently,~~~~}
\{ x  \in  \mathbb{R} \mid \exists p \in \mathbb{Z}  \exists q \in \mathbb{Z}^+ ( p =  x \cdot q) \}
\]
{\it Extra practice}: Use the definition of set equality to prove that the definitions above  give the same set.

\end{document}
\input{../activity-snippets/sets-numbers-subsets.tex}
\input{../activity-snippets/proof-by-contradiction-irrational.tex}

\newpage
\subsection*{Review}
\begin{enumerate}
\item \hspace{1in}\\ \input{../activity-snippets/quiz-calculating-gcd.tex}
\newpage
\item \hspace{1in}\\ \input{../activity-snippets/quiz-odd-even-proofs.tex}
\end{enumerate}

\newpage

\section*{Monday November 15}

\input{../activity-snippets/sets-numbers-subsets.tex}
\documentclass[12pt, oneside]{article}

\usepackage[letterpaper, scale=0.89, centering]{geometry}
\usepackage{fancyhdr}
\setlength{\parindent}{0em}
\setlength{\parskip}{1em}

\pagestyle{fancy}
\fancyhf{}
\renewcommand{\headrulewidth}{0pt}
\rfoot{\href{https://creativecommons.org/licenses/by-nc-sa/2.0/}{CC BY-NC-SA 2.0} Version \today~(\thepage)}

\usepackage{amssymb,amsmath,pifont,amsfonts,comment,enumerate,enumitem}
\usepackage{currfile,xstring,hyperref,tabularx,graphicx,wasysym}
\usepackage[labelformat=empty]{caption}
\usepackage[dvipsnames,table]{xcolor}
\usepackage{multicol,multirow,array,listings,tabularx,lastpage,textcomp,booktabs}

% NOTE(joe): This environment is credit @pnpo (https://tex.stackexchange.com/a/218450)
\lstnewenvironment{algorithm}[1][] %defines the algorithm listing environment
{   
    \lstset{ %this is the stype
        mathescape=true,
        frame=tB,
        numbers=left, 
        numberstyle=\tiny,
        basicstyle=\rmfamily\scriptsize, 
        keywordstyle=\color{black}\bfseries,
        keywords={,procedure, div, for, to, input, output, return, datatype, function, in, if, else, foreach, while, begin, end, }
        numbers=left,
        xleftmargin=.04\textwidth,
        #1
    }
}
{}
\lstnewenvironment{java}[1][]
{   
    \lstset{
        language=java,
        mathescape=true,
        frame=tB,
        numbers=left, 
        numberstyle=\tiny,
        basicstyle=\ttfamily\scriptsize, 
        keywordstyle=\color{black}\bfseries,
        keywords={, int, double, for, return, if, else, while, }
        numbers=left,
        xleftmargin=.04\textwidth,
        #1
    }
}
{}

\newcommand\abs[1]{\lvert~#1~\rvert}
\newcommand{\st}{\mid}

\newcommand{\A}[0]{\texttt{A}}
\newcommand{\C}[0]{\texttt{C}}
\newcommand{\G}[0]{\texttt{G}}
\newcommand{\U}[0]{\texttt{U}}

\newcommand{\cmark}{\ding{51}}
\newcommand{\xmark}{\ding{55}}


\begin{document}
\begin{flushright}
    \StrBefore{\currfilename}{.}
\end{flushright}{\bf Definition}: A {\bf finite} set is one whose distinct elements can be counted by a natural number.

\end{document}
\input{../activity-snippets/cardinality-motivation.tex}
\input{../activity-snippets/cardinality-rationale-for-functions.tex}
\input{../activity-snippets/musical-chairs-analogy.tex}
\newpage
\input{../activity-snippets/well-defined-functions.tex}
\documentclass[12pt, oneside]{article}

\usepackage[letterpaper, scale=0.89, centering]{geometry}
\usepackage{fancyhdr}
\setlength{\parindent}{0em}
\setlength{\parskip}{1em}

\pagestyle{fancy}
\fancyhf{}
\renewcommand{\headrulewidth}{0pt}
\rfoot{\href{https://creativecommons.org/licenses/by-nc-sa/2.0/}{CC BY-NC-SA 2.0} Version \today~(\thepage)}

\usepackage{amssymb,amsmath,pifont,amsfonts,comment,enumerate,enumitem}
\usepackage{currfile,xstring,hyperref,tabularx,graphicx,wasysym}
\usepackage[labelformat=empty]{caption}
\usepackage[dvipsnames,table]{xcolor}
\usepackage{multicol,multirow,array,listings,tabularx,lastpage,textcomp,booktabs}

% NOTE(joe): This environment is credit @pnpo (https://tex.stackexchange.com/a/218450)
\lstnewenvironment{algorithm}[1][] %defines the algorithm listing environment
{   
    \lstset{ %this is the stype
        mathescape=true,
        frame=tB,
        numbers=left, 
        numberstyle=\tiny,
        basicstyle=\rmfamily\scriptsize, 
        keywordstyle=\color{black}\bfseries,
        keywords={,procedure, div, for, to, input, output, return, datatype, function, in, if, else, foreach, while, begin, end, }
        numbers=left,
        xleftmargin=.04\textwidth,
        #1
    }
}
{}
\lstnewenvironment{java}[1][]
{   
    \lstset{
        language=java,
        mathescape=true,
        frame=tB,
        numbers=left, 
        numberstyle=\tiny,
        basicstyle=\ttfamily\scriptsize, 
        keywordstyle=\color{black}\bfseries,
        keywords={, int, double, for, return, if, else, while, }
        numbers=left,
        xleftmargin=.04\textwidth,
        #1
    }
}
{}

\newcommand\abs[1]{\lvert~#1~\rvert}
\newcommand{\st}{\mid}

\newcommand{\A}[0]{\texttt{A}}
\newcommand{\C}[0]{\texttt{C}}
\newcommand{\G}[0]{\texttt{G}}
\newcommand{\U}[0]{\texttt{U}}

\newcommand{\cmark}{\ding{51}}
\newcommand{\xmark}{\ding{55}}


\begin{document}
\begin{flushright}
    \StrBefore{\currfilename}{.}
\end{flushright}{\bf Definition} : A function $f: D  \to C$ is {\bf one-to-one} (or  injective) 
means for every $a,b$ in the domain $D$, 
if $f(a) = f(b)$ then  $a=b$.
Formally, $f: D  \to  C$ is  one-to-one  means $\underline{\phantom{\forall a \in D \forall b \in D ~(f(a) = f(b) \to a = b)}}$.

\end{document}
\input{../activity-snippets/injective-functions-visually.tex}
\documentclass[12pt, oneside]{article}

\usepackage[letterpaper, scale=0.89, centering]{geometry}
\usepackage{fancyhdr}
\setlength{\parindent}{0em}
\setlength{\parskip}{1em}

\pagestyle{fancy}
\fancyhf{}
\renewcommand{\headrulewidth}{0pt}
\rfoot{\href{https://creativecommons.org/licenses/by-nc-sa/2.0/}{CC BY-NC-SA 2.0} Version \today~(\thepage)}

\usepackage{amssymb,amsmath,pifont,amsfonts,comment,enumerate,enumitem}
\usepackage{currfile,xstring,hyperref,tabularx,graphicx,wasysym}
\usepackage[labelformat=empty]{caption}
\usepackage[dvipsnames,table]{xcolor}
\usepackage{multicol,multirow,array,listings,tabularx,lastpage,textcomp,booktabs}

% NOTE(joe): This environment is credit @pnpo (https://tex.stackexchange.com/a/218450)
\lstnewenvironment{algorithm}[1][] %defines the algorithm listing environment
{   
    \lstset{ %this is the stype
        mathescape=true,
        frame=tB,
        numbers=left, 
        numberstyle=\tiny,
        basicstyle=\rmfamily\scriptsize, 
        keywordstyle=\color{black}\bfseries,
        keywords={,procedure, div, for, to, input, output, return, datatype, function, in, if, else, foreach, while, begin, end, }
        numbers=left,
        xleftmargin=.04\textwidth,
        #1
    }
}
{}
\lstnewenvironment{java}[1][]
{   
    \lstset{
        language=java,
        mathescape=true,
        frame=tB,
        numbers=left, 
        numberstyle=\tiny,
        basicstyle=\ttfamily\scriptsize, 
        keywordstyle=\color{black}\bfseries,
        keywords={, int, double, for, return, if, else, while, }
        numbers=left,
        xleftmargin=.04\textwidth,
        #1
    }
}
{}

\newcommand\abs[1]{\lvert~#1~\rvert}
\newcommand{\st}{\mid}

\newcommand{\A}[0]{\texttt{A}}
\newcommand{\C}[0]{\texttt{C}}
\newcommand{\G}[0]{\texttt{G}}
\newcommand{\U}[0]{\texttt{U}}

\newcommand{\cmark}{\ding{51}}
\newcommand{\xmark}{\ding{55}}


\begin{document}
\begin{flushright}
    \StrBefore{\currfilename}{.}
\end{flushright}{\bf Definition}:  For nonempty sets $A, B$, we say that {\bf the  cardinality of $A$ is  no  bigger than the cardinality of $B$}, 
and write $|A| \leq |B|$, to mean there is a  one-to-one function  with domain $A$  and codomain $B$.
Also, we define $|\emptyset| \leq |B|$ for all sets $B$.
\end{document}
\input{../activity-snippets/injective-cardinality-musical-chairs.tex}
\input{../activity-snippets/rna-injective-cardinality.tex}
\documentclass[12pt, oneside]{article}

\usepackage[letterpaper, scale=0.89, centering]{geometry}
\usepackage{fancyhdr}
\setlength{\parindent}{0em}
\setlength{\parskip}{1em}

\pagestyle{fancy}
\fancyhf{}
\renewcommand{\headrulewidth}{0pt}
\rfoot{\href{https://creativecommons.org/licenses/by-nc-sa/2.0/}{CC BY-NC-SA 2.0} Version \today~(\thepage)}

\usepackage{amssymb,amsmath,pifont,amsfonts,comment,enumerate,enumitem}
\usepackage{currfile,xstring,hyperref,tabularx,graphicx,wasysym}
\usepackage[labelformat=empty]{caption}
\usepackage[dvipsnames,table]{xcolor}
\usepackage{multicol,multirow,array,listings,tabularx,lastpage,textcomp,booktabs}

% NOTE(joe): This environment is credit @pnpo (https://tex.stackexchange.com/a/218450)
\lstnewenvironment{algorithm}[1][] %defines the algorithm listing environment
{   
    \lstset{ %this is the stype
        mathescape=true,
        frame=tB,
        numbers=left, 
        numberstyle=\tiny,
        basicstyle=\rmfamily\scriptsize, 
        keywordstyle=\color{black}\bfseries,
        keywords={,procedure, div, for, to, input, output, return, datatype, function, in, if, else, foreach, while, begin, end, }
        numbers=left,
        xleftmargin=.04\textwidth,
        #1
    }
}
{}
\lstnewenvironment{java}[1][]
{   
    \lstset{
        language=java,
        mathescape=true,
        frame=tB,
        numbers=left, 
        numberstyle=\tiny,
        basicstyle=\ttfamily\scriptsize, 
        keywordstyle=\color{black}\bfseries,
        keywords={, int, double, for, return, if, else, while, }
        numbers=left,
        xleftmargin=.04\textwidth,
        #1
    }
}
{}

\newcommand\abs[1]{\lvert~#1~\rvert}
\newcommand{\st}{\mid}

\newcommand{\A}[0]{\texttt{A}}
\newcommand{\C}[0]{\texttt{C}}
\newcommand{\G}[0]{\texttt{G}}
\newcommand{\U}[0]{\texttt{U}}

\newcommand{\cmark}{\ding{51}}
\newcommand{\xmark}{\ding{55}}


\begin{document}
\begin{flushright}
    \StrBefore{\currfilename}{.}
\end{flushright}{\bf Definition}: A function $f: D  \to C$ is {\bf onto} (or  surjective) means for every $b$ in the codomain, 
there  is an element $a$ in the domain with  $f(a) = b$.
Formally, $f: D  \to  C$ is  onto  means $\underline{\phantom{\forall b \in C  \exists a \in D ( f(a) = b)}}$.

\end{document}
\input{../activity-snippets/surjective-functions-visually.tex}
\documentclass[12pt, oneside]{article}

\usepackage[letterpaper, scale=0.89, centering]{geometry}
\usepackage{fancyhdr}
\setlength{\parindent}{0em}
\setlength{\parskip}{1em}

\pagestyle{fancy}
\fancyhf{}
\renewcommand{\headrulewidth}{0pt}
\rfoot{\href{https://creativecommons.org/licenses/by-nc-sa/2.0/}{CC BY-NC-SA 2.0} Version \today~(\thepage)}

\usepackage{amssymb,amsmath,pifont,amsfonts,comment,enumerate,enumitem}
\usepackage{currfile,xstring,hyperref,tabularx,graphicx,wasysym}
\usepackage[labelformat=empty]{caption}
\usepackage[dvipsnames,table]{xcolor}
\usepackage{multicol,multirow,array,listings,tabularx,lastpage,textcomp,booktabs}

% NOTE(joe): This environment is credit @pnpo (https://tex.stackexchange.com/a/218450)
\lstnewenvironment{algorithm}[1][] %defines the algorithm listing environment
{   
    \lstset{ %this is the stype
        mathescape=true,
        frame=tB,
        numbers=left, 
        numberstyle=\tiny,
        basicstyle=\rmfamily\scriptsize, 
        keywordstyle=\color{black}\bfseries,
        keywords={,procedure, div, for, to, input, output, return, datatype, function, in, if, else, foreach, while, begin, end, }
        numbers=left,
        xleftmargin=.04\textwidth,
        #1
    }
}
{}
\lstnewenvironment{java}[1][]
{   
    \lstset{
        language=java,
        mathescape=true,
        frame=tB,
        numbers=left, 
        numberstyle=\tiny,
        basicstyle=\ttfamily\scriptsize, 
        keywordstyle=\color{black}\bfseries,
        keywords={, int, double, for, return, if, else, while, }
        numbers=left,
        xleftmargin=.04\textwidth,
        #1
    }
}
{}

\newcommand\abs[1]{\lvert~#1~\rvert}
\newcommand{\st}{\mid}

\newcommand{\A}[0]{\texttt{A}}
\newcommand{\C}[0]{\texttt{C}}
\newcommand{\G}[0]{\texttt{G}}
\newcommand{\U}[0]{\texttt{U}}

\newcommand{\cmark}{\ding{51}}
\newcommand{\xmark}{\ding{55}}


\begin{document}
\begin{flushright}
    \StrBefore{\currfilename}{.}
\end{flushright}{\bf Definition}:  For nonempty sets $A, B$, we say that {\bf the  cardinality of $A$ is  no  smaller than 
the cardinality of  $B$}, and 
write $|A| \geq |B|$, to mean there is an onto function  with domain $A$  and codomain $B$.
Also, we define $|A| \geq |\emptyset|$ for all sets $A$.

\end{document}
\input{../activity-snippets/surjective-cardinality-musical-chairs.tex}
\newpage
\input{../activity-snippets/rna-surjective-cardinality.tex}
\documentclass[12pt, oneside]{article}

\usepackage[letterpaper, scale=0.89, centering]{geometry}
\usepackage{fancyhdr}
\setlength{\parindent}{0em}
\setlength{\parskip}{1em}

\pagestyle{fancy}
\fancyhf{}
\renewcommand{\headrulewidth}{0pt}
\rfoot{\href{https://creativecommons.org/licenses/by-nc-sa/2.0/}{CC BY-NC-SA 2.0} Version \today~(\thepage)}

\usepackage{amssymb,amsmath,pifont,amsfonts,comment,enumerate,enumitem}
\usepackage{currfile,xstring,hyperref,tabularx,graphicx,wasysym}
\usepackage[labelformat=empty]{caption}
\usepackage[dvipsnames,table]{xcolor}
\usepackage{multicol,multirow,array,listings,tabularx,lastpage,textcomp,booktabs}

% NOTE(joe): This environment is credit @pnpo (https://tex.stackexchange.com/a/218450)
\lstnewenvironment{algorithm}[1][] %defines the algorithm listing environment
{   
    \lstset{ %this is the stype
        mathescape=true,
        frame=tB,
        numbers=left, 
        numberstyle=\tiny,
        basicstyle=\rmfamily\scriptsize, 
        keywordstyle=\color{black}\bfseries,
        keywords={,procedure, div, for, to, input, output, return, datatype, function, in, if, else, foreach, while, begin, end, }
        numbers=left,
        xleftmargin=.04\textwidth,
        #1
    }
}
{}
\lstnewenvironment{java}[1][]
{   
    \lstset{
        language=java,
        mathescape=true,
        frame=tB,
        numbers=left, 
        numberstyle=\tiny,
        basicstyle=\ttfamily\scriptsize, 
        keywordstyle=\color{black}\bfseries,
        keywords={, int, double, for, return, if, else, while, }
        numbers=left,
        xleftmargin=.04\textwidth,
        #1
    }
}
{}

\newcommand\abs[1]{\lvert~#1~\rvert}
\newcommand{\st}{\mid}

\newcommand{\A}[0]{\texttt{A}}
\newcommand{\C}[0]{\texttt{C}}
\newcommand{\G}[0]{\texttt{G}}
\newcommand{\U}[0]{\texttt{U}}

\newcommand{\cmark}{\ding{51}}
\newcommand{\xmark}{\ding{55}}


\begin{document}
\begin{flushright}
    \StrBefore{\currfilename}{.}
\end{flushright}{\bf Definition} : A function $f: D  \to C$ is a {\bf bijection} means that it is both 
one-to-one  and onto. The {\bf inverse} of a  bijection $f: D  \to  C$ is 
the function $g: C  \to  D$  such that $g(b) = a$ iff  $f(a) =  b$.

\end{document}
\subsection*{Cardinality of sets}
\documentclass[12pt, oneside]{article}

\usepackage[letterpaper, scale=0.89, centering]{geometry}
\usepackage{fancyhdr}
\setlength{\parindent}{0em}
\setlength{\parskip}{1em}

\pagestyle{fancy}
\fancyhf{}
\renewcommand{\headrulewidth}{0pt}
\rfoot{\href{https://creativecommons.org/licenses/by-nc-sa/2.0/}{CC BY-NC-SA 2.0} Version \today~(\thepage)}

\usepackage{amssymb,amsmath,pifont,amsfonts,comment,enumerate,enumitem}
\usepackage{currfile,xstring,hyperref,tabularx,graphicx,wasysym}
\usepackage[labelformat=empty]{caption}
\usepackage[dvipsnames,table]{xcolor}
\usepackage{multicol,multirow,array,listings,tabularx,lastpage,textcomp,booktabs}

% NOTE(joe): This environment is credit @pnpo (https://tex.stackexchange.com/a/218450)
\lstnewenvironment{algorithm}[1][] %defines the algorithm listing environment
{   
    \lstset{ %this is the stype
        mathescape=true,
        frame=tB,
        numbers=left, 
        numberstyle=\tiny,
        basicstyle=\rmfamily\scriptsize, 
        keywordstyle=\color{black}\bfseries,
        keywords={,procedure, div, for, to, input, output, return, datatype, function, in, if, else, foreach, while, begin, end, }
        numbers=left,
        xleftmargin=.04\textwidth,
        #1
    }
}
{}
\lstnewenvironment{java}[1][]
{   
    \lstset{
        language=java,
        mathescape=true,
        frame=tB,
        numbers=left, 
        numberstyle=\tiny,
        basicstyle=\ttfamily\scriptsize, 
        keywordstyle=\color{black}\bfseries,
        keywords={, int, double, for, return, if, else, while, }
        numbers=left,
        xleftmargin=.04\textwidth,
        #1
    }
}
{}

\newcommand\abs[1]{\lvert~#1~\rvert}
\newcommand{\st}{\mid}

\newcommand{\A}[0]{\texttt{A}}
\newcommand{\C}[0]{\texttt{C}}
\newcommand{\G}[0]{\texttt{G}}
\newcommand{\U}[0]{\texttt{U}}

\newcommand{\cmark}{\ding{51}}
\newcommand{\xmark}{\ding{55}}


\begin{document}
\begin{flushright}
    \StrBefore{\currfilename}{.}
\end{flushright}For nonempty sets $A, B$ we say
\begin{align*}
|A| \leq |B| &\text{ means there is a one-to-one function with domain $A$, codomain $B$} \\
|A| \geq |B| &\text{ means there is an onto function with domain $A$, codomain $B$} \\
|A| = |B| &\text{ means there is a bijection with domain $A$, codomain $B$}
\end{align*}
For all sets $A$, we say $|A| = |\emptyset|$, $|\emptyset| = |A|$ if and only if $A = \emptyset$.
\end{document}
\input{../activity-snippets/cardinality-caution.tex}
\newpage
\section*{Review}
\begin{enumerate}
    \item \hspace{1in}\\ \input{../activity-snippets/quiz-finite-sets.tex}
    \item \hspace{1in}\\ \input{../activity-snippets/quiz-injective-surjective.tex}
    \newpage
    \item \hspace{1in}\\ \input{../activity-snippets/quiz-cardinality-witnessing-functions-q1.tex} 
\end{enumerate}
\newpage

\section*{Wednesday November 17}
\input{../activity-snippets/cardinality-properties.tex}
\vspace{100pt}
\input{../activity-snippets/cantor-schroder-bernstein-theorem.tex}
\newpage
\documentclass[12pt, oneside]{article}

\usepackage[letterpaper, scale=0.89, centering]{geometry}
\usepackage{fancyhdr}
\setlength{\parindent}{0em}
\setlength{\parskip}{1em}

\pagestyle{fancy}
\fancyhf{}
\renewcommand{\headrulewidth}{0pt}
\rfoot{\href{https://creativecommons.org/licenses/by-nc-sa/2.0/}{CC BY-NC-SA 2.0} Version \today~(\thepage)}

\usepackage{amssymb,amsmath,pifont,amsfonts,comment,enumerate,enumitem}
\usepackage{currfile,xstring,hyperref,tabularx,graphicx,wasysym}
\usepackage[labelformat=empty]{caption}
\usepackage[dvipsnames,table]{xcolor}
\usepackage{multicol,multirow,array,listings,tabularx,lastpage,textcomp,booktabs}

% NOTE(joe): This environment is credit @pnpo (https://tex.stackexchange.com/a/218450)
\lstnewenvironment{algorithm}[1][] %defines the algorithm listing environment
{   
    \lstset{ %this is the stype
        mathescape=true,
        frame=tB,
        numbers=left, 
        numberstyle=\tiny,
        basicstyle=\rmfamily\scriptsize, 
        keywordstyle=\color{black}\bfseries,
        keywords={,procedure, div, for, to, input, output, return, datatype, function, in, if, else, foreach, while, begin, end, }
        numbers=left,
        xleftmargin=.04\textwidth,
        #1
    }
}
{}
\lstnewenvironment{java}[1][]
{   
    \lstset{
        language=java,
        mathescape=true,
        frame=tB,
        numbers=left, 
        numberstyle=\tiny,
        basicstyle=\ttfamily\scriptsize, 
        keywordstyle=\color{black}\bfseries,
        keywords={, int, double, for, return, if, else, while, }
        numbers=left,
        xleftmargin=.04\textwidth,
        #1
    }
}
{}

\newcommand\abs[1]{\lvert~#1~\rvert}
\newcommand{\st}{\mid}

\newcommand{\A}[0]{\texttt{A}}
\newcommand{\C}[0]{\texttt{C}}
\newcommand{\G}[0]{\texttt{G}}
\newcommand{\U}[0]{\texttt{U}}

\newcommand{\cmark}{\ding{51}}
\newcommand{\xmark}{\ding{55}}


\begin{document}
\begin{flushright}
    \StrBefore{\currfilename}{.}
\end{flushright}{\bf Definition}: A set $A$ is {\bf countably infinite} means it is the 
same size as $\mathbb{N}$.

\end{document}
\input{../activity-snippets/countably-infinite-examples-sets-of-numbers.tex}
\input{../activity-snippets/countably-infinite-examples-other-sets.tex}
\newpage
\subsection*{Review}
\begin{enumerate}
    \item \input{../activity-snippets/quiz-cardinality-witnessing-functions-q2.tex} 
    \item \hspace{1in}\\ \input{../activity-snippets/quiz-rationals-proofs.tex}
\end{enumerate}

\newpage
\section*{Friday November 19}
\subsection*{Cardinality categories}
\input{../activity-snippets/cardinality-categories.tex}
\input{../activity-snippets/cardinality-countability-lemmas.tex}
\newpage
\subsection*{Are there always *bigger* sets?}
\input{../activity-snippets/cardinality-power-sets.tex}
\newpage
\subsection*{Countable vs.\ uncountable: sets of numbers}
\input{../activity-snippets/cardinality-rationals-reals.tex}
\subsection*{Other examples of uncountable sets}
\input{../activity-snippets/cardinality-uncountable-examples.tex}
\newpage
\subsection*{Review}
\begin{enumerate}
    \item \hspace{1in}\\ \input{../activity-snippets/quiz-diagonalization.tex}
    \item \hspace{1in}\\ \input{../activity-snippets/quiz-cardinality-classifying.tex}
\end{enumerate}
\newpage

\section*{Monday November 22}

\documentclass[12pt, oneside]{article}

\usepackage[letterpaper, scale=0.89, centering]{geometry}
\usepackage{fancyhdr}
\setlength{\parindent}{0em}
\setlength{\parskip}{1em}

\pagestyle{fancy}
\fancyhf{}
\renewcommand{\headrulewidth}{0pt}
\rfoot{\href{https://creativecommons.org/licenses/by-nc-sa/2.0/}{CC BY-NC-SA 2.0} Version \today~(\thepage)}

\usepackage{amssymb,amsmath,pifont,amsfonts,comment,enumerate,enumitem}
\usepackage{currfile,xstring,hyperref,tabularx,graphicx,wasysym}
\usepackage[labelformat=empty]{caption}
\usepackage[dvipsnames,table]{xcolor}
\usepackage{multicol,multirow,array,listings,tabularx,lastpage,textcomp,booktabs}

% NOTE(joe): This environment is credit @pnpo (https://tex.stackexchange.com/a/218450)
\lstnewenvironment{algorithm}[1][] %defines the algorithm listing environment
{   
    \lstset{ %this is the stype
        mathescape=true,
        frame=tB,
        numbers=left, 
        numberstyle=\tiny,
        basicstyle=\rmfamily\scriptsize, 
        keywordstyle=\color{black}\bfseries,
        keywords={,procedure, div, for, to, input, output, return, datatype, function, in, if, else, foreach, while, begin, end, }
        numbers=left,
        xleftmargin=.04\textwidth,
        #1
    }
}
{}
\lstnewenvironment{java}[1][]
{   
    \lstset{
        language=java,
        mathescape=true,
        frame=tB,
        numbers=left, 
        numberstyle=\tiny,
        basicstyle=\ttfamily\scriptsize, 
        keywordstyle=\color{black}\bfseries,
        keywords={, int, double, for, return, if, else, while, }
        numbers=left,
        xleftmargin=.04\textwidth,
        #1
    }
}
{}

\newcommand\abs[1]{\lvert~#1~\rvert}
\newcommand{\st}{\mid}

\newcommand{\A}[0]{\texttt{A}}
\newcommand{\C}[0]{\texttt{C}}
\newcommand{\G}[0]{\texttt{G}}
\newcommand{\U}[0]{\texttt{U}}

\newcommand{\cmark}{\ding{51}}
\newcommand{\xmark}{\ding{55}}


\begin{document}
\begin{flushright}
    \StrBefore{\currfilename}{.}
\end{flushright}{\bf Definition}: When $A$ and $B$ are sets, we say any subset of $A \times B$ is a {\bf binary relation}. 
A relation $R$ can also be represented as
\begin{itemize}
\item A function $f_{TF} : A \times B \to \{T, F\}$
where, for $a \in A$ and $b \in B$, $f_{TF}(~(a,b)~) = 
\begin{cases} 
    T \qquad&\text{when } (a,b) \in R \\
    F \qquad&\text{when } (a,b) \notin R
\end{cases}$
\item A function $f_{\mathcal{P}} : A   \to \mathcal{P}(B)$ where, for $a \in A$, 
$f_{\mathcal{P}}( a ) = \{ b \in B ~|~ (a,b) \in R \}$
\end{itemize}
When $A$ is a set, we say any subset of $A \times A$ is a (binary) {\bf relation} on $A$.

\end{document}
\input{../activity-snippets/relations-as-graphs.tex}
\input{../activity-snippets/binary-relation-examples.tex}
\newpage
\documentclass[12pt, oneside]{article}

\usepackage[letterpaper, scale=0.89, centering]{geometry}
\usepackage{fancyhdr}
\setlength{\parindent}{0em}
\setlength{\parskip}{1em}

\pagestyle{fancy}
\fancyhf{}
\renewcommand{\headrulewidth}{0pt}
\rfoot{\href{https://creativecommons.org/licenses/by-nc-sa/2.0/}{CC BY-NC-SA 2.0} Version \today~(\thepage)}

\usepackage{amssymb,amsmath,pifont,amsfonts,comment,enumerate,enumitem}
\usepackage{currfile,xstring,hyperref,tabularx,graphicx,wasysym}
\usepackage[labelformat=empty]{caption}
\usepackage[dvipsnames,table]{xcolor}
\usepackage{multicol,multirow,array,listings,tabularx,lastpage,textcomp,booktabs}

% NOTE(joe): This environment is credit @pnpo (https://tex.stackexchange.com/a/218450)
\lstnewenvironment{algorithm}[1][] %defines the algorithm listing environment
{   
    \lstset{ %this is the stype
        mathescape=true,
        frame=tB,
        numbers=left, 
        numberstyle=\tiny,
        basicstyle=\rmfamily\scriptsize, 
        keywordstyle=\color{black}\bfseries,
        keywords={,procedure, div, for, to, input, output, return, datatype, function, in, if, else, foreach, while, begin, end, }
        numbers=left,
        xleftmargin=.04\textwidth,
        #1
    }
}
{}
\lstnewenvironment{java}[1][]
{   
    \lstset{
        language=java,
        mathescape=true,
        frame=tB,
        numbers=left, 
        numberstyle=\tiny,
        basicstyle=\ttfamily\scriptsize, 
        keywordstyle=\color{black}\bfseries,
        keywords={, int, double, for, return, if, else, while, }
        numbers=left,
        xleftmargin=.04\textwidth,
        #1
    }
}
{}

\newcommand\abs[1]{\lvert~#1~\rvert}
\newcommand{\st}{\mid}

\newcommand{\A}[0]{\texttt{A}}
\newcommand{\C}[0]{\texttt{C}}
\newcommand{\G}[0]{\texttt{G}}
\newcommand{\U}[0]{\texttt{U}}

\newcommand{\cmark}{\ding{51}}
\newcommand{\xmark}{\ding{55}}


\begin{document}
\begin{flushright}
    \StrBefore{\currfilename}{.}
\end{flushright}A relation $R$ on a set $A$ is called {\bf reflexive} 
means $(a, a) \in R$ for every element $a \in A$. 

\end{document}
\input{../activity-snippets/reflexive-relation-informally.tex}
\vfill
\documentclass[12pt, oneside]{article}

\usepackage[letterpaper, scale=0.89, centering]{geometry}
\usepackage{fancyhdr}
\setlength{\parindent}{0em}
\setlength{\parskip}{1em}

\pagestyle{fancy}
\fancyhf{}
\renewcommand{\headrulewidth}{0pt}
\rfoot{\href{https://creativecommons.org/licenses/by-nc-sa/2.0/}{CC BY-NC-SA 2.0} Version \today~(\thepage)}

\usepackage{amssymb,amsmath,pifont,amsfonts,comment,enumerate,enumitem}
\usepackage{currfile,xstring,hyperref,tabularx,graphicx,wasysym}
\usepackage[labelformat=empty]{caption}
\usepackage[dvipsnames,table]{xcolor}
\usepackage{multicol,multirow,array,listings,tabularx,lastpage,textcomp,booktabs}

% NOTE(joe): This environment is credit @pnpo (https://tex.stackexchange.com/a/218450)
\lstnewenvironment{algorithm}[1][] %defines the algorithm listing environment
{   
    \lstset{ %this is the stype
        mathescape=true,
        frame=tB,
        numbers=left, 
        numberstyle=\tiny,
        basicstyle=\rmfamily\scriptsize, 
        keywordstyle=\color{black}\bfseries,
        keywords={,procedure, div, for, to, input, output, return, datatype, function, in, if, else, foreach, while, begin, end, }
        numbers=left,
        xleftmargin=.04\textwidth,
        #1
    }
}
{}
\lstnewenvironment{java}[1][]
{   
    \lstset{
        language=java,
        mathescape=true,
        frame=tB,
        numbers=left, 
        numberstyle=\tiny,
        basicstyle=\ttfamily\scriptsize, 
        keywordstyle=\color{black}\bfseries,
        keywords={, int, double, for, return, if, else, while, }
        numbers=left,
        xleftmargin=.04\textwidth,
        #1
    }
}
{}

\newcommand\abs[1]{\lvert~#1~\rvert}
\newcommand{\st}{\mid}

\newcommand{\A}[0]{\texttt{A}}
\newcommand{\C}[0]{\texttt{C}}
\newcommand{\G}[0]{\texttt{G}}
\newcommand{\U}[0]{\texttt{U}}

\newcommand{\cmark}{\ding{51}}
\newcommand{\xmark}{\ding{55}}


\begin{document}
\begin{flushright}
    \StrBefore{\currfilename}{.}
\end{flushright}A relation $R$ on a set $A$ is called {\bf symmetric} means 
$(b, a) \in R$ whenever $(a, b) \in R$, for all $a, b \in A$. 

\end{document}
\input{../activity-snippets/symmetric-relation-informally.tex}
\vfill
\documentclass[12pt, oneside]{article}

\usepackage[letterpaper, scale=0.89, centering]{geometry}
\usepackage{fancyhdr}
\setlength{\parindent}{0em}
\setlength{\parskip}{1em}

\pagestyle{fancy}
\fancyhf{}
\renewcommand{\headrulewidth}{0pt}
\rfoot{\href{https://creativecommons.org/licenses/by-nc-sa/2.0/}{CC BY-NC-SA 2.0} Version \today~(\thepage)}

\usepackage{amssymb,amsmath,pifont,amsfonts,comment,enumerate,enumitem}
\usepackage{currfile,xstring,hyperref,tabularx,graphicx,wasysym}
\usepackage[labelformat=empty]{caption}
\usepackage[dvipsnames,table]{xcolor}
\usepackage{multicol,multirow,array,listings,tabularx,lastpage,textcomp,booktabs}

% NOTE(joe): This environment is credit @pnpo (https://tex.stackexchange.com/a/218450)
\lstnewenvironment{algorithm}[1][] %defines the algorithm listing environment
{   
    \lstset{ %this is the stype
        mathescape=true,
        frame=tB,
        numbers=left, 
        numberstyle=\tiny,
        basicstyle=\rmfamily\scriptsize, 
        keywordstyle=\color{black}\bfseries,
        keywords={,procedure, div, for, to, input, output, return, datatype, function, in, if, else, foreach, while, begin, end, }
        numbers=left,
        xleftmargin=.04\textwidth,
        #1
    }
}
{}
\lstnewenvironment{java}[1][]
{   
    \lstset{
        language=java,
        mathescape=true,
        frame=tB,
        numbers=left, 
        numberstyle=\tiny,
        basicstyle=\ttfamily\scriptsize, 
        keywordstyle=\color{black}\bfseries,
        keywords={, int, double, for, return, if, else, while, }
        numbers=left,
        xleftmargin=.04\textwidth,
        #1
    }
}
{}

\newcommand\abs[1]{\lvert~#1~\rvert}
\newcommand{\st}{\mid}

\newcommand{\A}[0]{\texttt{A}}
\newcommand{\C}[0]{\texttt{C}}
\newcommand{\G}[0]{\texttt{G}}
\newcommand{\U}[0]{\texttt{U}}

\newcommand{\cmark}{\ding{51}}
\newcommand{\xmark}{\ding{55}}


\begin{document}
\begin{flushright}
    \StrBefore{\currfilename}{.}
\end{flushright}A relation $R$ on a set $A$ is called {\bf transitive} means 
whenever $(a, b) \in R$ and $(b, c) \in R$, then $(a, c) \in R$, for all $a, b, c \in A$.

\end{document}
\input{../activity-snippets/transitive-relation-informally.tex}
\vfill
\documentclass[12pt, oneside]{article}

\usepackage[letterpaper, scale=0.89, centering]{geometry}
\usepackage{fancyhdr}
\setlength{\parindent}{0em}
\setlength{\parskip}{1em}

\pagestyle{fancy}
\fancyhf{}
\renewcommand{\headrulewidth}{0pt}
\rfoot{\href{https://creativecommons.org/licenses/by-nc-sa/2.0/}{CC BY-NC-SA 2.0} Version \today~(\thepage)}

\usepackage{amssymb,amsmath,pifont,amsfonts,comment,enumerate,enumitem}
\usepackage{currfile,xstring,hyperref,tabularx,graphicx,wasysym}
\usepackage[labelformat=empty]{caption}
\usepackage[dvipsnames,table]{xcolor}
\usepackage{multicol,multirow,array,listings,tabularx,lastpage,textcomp,booktabs}

% NOTE(joe): This environment is credit @pnpo (https://tex.stackexchange.com/a/218450)
\lstnewenvironment{algorithm}[1][] %defines the algorithm listing environment
{   
    \lstset{ %this is the stype
        mathescape=true,
        frame=tB,
        numbers=left, 
        numberstyle=\tiny,
        basicstyle=\rmfamily\scriptsize, 
        keywordstyle=\color{black}\bfseries,
        keywords={,procedure, div, for, to, input, output, return, datatype, function, in, if, else, foreach, while, begin, end, }
        numbers=left,
        xleftmargin=.04\textwidth,
        #1
    }
}
{}
\lstnewenvironment{java}[1][]
{   
    \lstset{
        language=java,
        mathescape=true,
        frame=tB,
        numbers=left, 
        numberstyle=\tiny,
        basicstyle=\ttfamily\scriptsize, 
        keywordstyle=\color{black}\bfseries,
        keywords={, int, double, for, return, if, else, while, }
        numbers=left,
        xleftmargin=.04\textwidth,
        #1
    }
}
{}

\newcommand\abs[1]{\lvert~#1~\rvert}
\newcommand{\st}{\mid}

\newcommand{\A}[0]{\texttt{A}}
\newcommand{\C}[0]{\texttt{C}}
\newcommand{\G}[0]{\texttt{G}}
\newcommand{\U}[0]{\texttt{U}}

\newcommand{\cmark}{\ding{51}}
\newcommand{\xmark}{\ding{55}}


\begin{document}
\begin{flushright}
    \StrBefore{\currfilename}{.}
\end{flushright}A relation $R$ on a set $A$ is called {\bf antisymmetric} means 
$\forall a \in A ~\forall b \in A~\left(~\left( ~(a,b) \in R \land (b,a) \in R ~\right) \to a=b~\right)$

\end{document}
\input{../activity-snippets/antisymmetric-relation-informally.tex}
\vfill
\newpage
\input{../activity-snippets/binary-relation-properties-examples.tex}
\newpage
\documentclass[12pt, oneside]{article}

\usepackage[letterpaper, scale=0.89, centering]{geometry}
\usepackage{fancyhdr}
\setlength{\parindent}{0em}
\setlength{\parskip}{1em}

\pagestyle{fancy}
\fancyhf{}
\renewcommand{\headrulewidth}{0pt}
\rfoot{\href{https://creativecommons.org/licenses/by-nc-sa/2.0/}{CC BY-NC-SA 2.0} Version \today~(\thepage)}

\usepackage{amssymb,amsmath,pifont,amsfonts,comment,enumerate,enumitem}
\usepackage{currfile,xstring,hyperref,tabularx,graphicx,wasysym}
\usepackage[labelformat=empty]{caption}
\usepackage[dvipsnames,table]{xcolor}
\usepackage{multicol,multirow,array,listings,tabularx,lastpage,textcomp,booktabs}

% NOTE(joe): This environment is credit @pnpo (https://tex.stackexchange.com/a/218450)
\lstnewenvironment{algorithm}[1][] %defines the algorithm listing environment
{   
    \lstset{ %this is the stype
        mathescape=true,
        frame=tB,
        numbers=left, 
        numberstyle=\tiny,
        basicstyle=\rmfamily\scriptsize, 
        keywordstyle=\color{black}\bfseries,
        keywords={,procedure, div, for, to, input, output, return, datatype, function, in, if, else, foreach, while, begin, end, }
        numbers=left,
        xleftmargin=.04\textwidth,
        #1
    }
}
{}
\lstnewenvironment{java}[1][]
{   
    \lstset{
        language=java,
        mathescape=true,
        frame=tB,
        numbers=left, 
        numberstyle=\tiny,
        basicstyle=\ttfamily\scriptsize, 
        keywordstyle=\color{black}\bfseries,
        keywords={, int, double, for, return, if, else, while, }
        numbers=left,
        xleftmargin=.04\textwidth,
        #1
    }
}
{}

\newcommand\abs[1]{\lvert~#1~\rvert}
\newcommand{\st}{\mid}

\newcommand{\A}[0]{\texttt{A}}
\newcommand{\C}[0]{\texttt{C}}
\newcommand{\G}[0]{\texttt{G}}
\newcommand{\U}[0]{\texttt{U}}

\newcommand{\cmark}{\ding{51}}
\newcommand{\xmark}{\ding{55}}


\begin{document}
\begin{flushright}
    \StrBefore{\currfilename}{.}
\end{flushright}A relation is an {\bf equivalence relation} means it is reflexive, symmetric, and transitive.
\end{document}
\documentclass[12pt, oneside]{article}

\usepackage[letterpaper, scale=0.89, centering]{geometry}
\usepackage{fancyhdr}
\setlength{\parindent}{0em}
\setlength{\parskip}{1em}

\pagestyle{fancy}
\fancyhf{}
\renewcommand{\headrulewidth}{0pt}
\rfoot{\href{https://creativecommons.org/licenses/by-nc-sa/2.0/}{CC BY-NC-SA 2.0} Version \today~(\thepage)}

\usepackage{amssymb,amsmath,pifont,amsfonts,comment,enumerate,enumitem}
\usepackage{currfile,xstring,hyperref,tabularx,graphicx,wasysym}
\usepackage[labelformat=empty]{caption}
\usepackage[dvipsnames,table]{xcolor}
\usepackage{multicol,multirow,array,listings,tabularx,lastpage,textcomp,booktabs}

% NOTE(joe): This environment is credit @pnpo (https://tex.stackexchange.com/a/218450)
\lstnewenvironment{algorithm}[1][] %defines the algorithm listing environment
{   
    \lstset{ %this is the stype
        mathescape=true,
        frame=tB,
        numbers=left, 
        numberstyle=\tiny,
        basicstyle=\rmfamily\scriptsize, 
        keywordstyle=\color{black}\bfseries,
        keywords={,procedure, div, for, to, input, output, return, datatype, function, in, if, else, foreach, while, begin, end, }
        numbers=left,
        xleftmargin=.04\textwidth,
        #1
    }
}
{}
\lstnewenvironment{java}[1][]
{   
    \lstset{
        language=java,
        mathescape=true,
        frame=tB,
        numbers=left, 
        numberstyle=\tiny,
        basicstyle=\ttfamily\scriptsize, 
        keywordstyle=\color{black}\bfseries,
        keywords={, int, double, for, return, if, else, while, }
        numbers=left,
        xleftmargin=.04\textwidth,
        #1
    }
}
{}

\newcommand\abs[1]{\lvert~#1~\rvert}
\newcommand{\st}{\mid}

\newcommand{\A}[0]{\texttt{A}}
\newcommand{\C}[0]{\texttt{C}}
\newcommand{\G}[0]{\texttt{G}}
\newcommand{\U}[0]{\texttt{U}}

\newcommand{\cmark}{\ding{51}}
\newcommand{\xmark}{\ding{55}}


\begin{document}
\begin{flushright}
    \StrBefore{\currfilename}{.}
\end{flushright}A relation is a {\bf partial ordering} (or partial order) means 
it is reflexive, antisymmetric, and transitive.
\end{document}
\documentclass[12pt, oneside]{article}

\usepackage[letterpaper, scale=0.89, centering]{geometry}
\usepackage{fancyhdr}
\setlength{\parindent}{0em}
\setlength{\parskip}{1em}

\pagestyle{fancy}
\fancyhf{}
\renewcommand{\headrulewidth}{0pt}
\rfoot{\href{https://creativecommons.org/licenses/by-nc-sa/2.0/}{CC BY-NC-SA 2.0} Version \today~(\thepage)}

\usepackage{amssymb,amsmath,pifont,amsfonts,comment,enumerate,enumitem}
\usepackage{currfile,xstring,hyperref,tabularx,graphicx,wasysym}
\usepackage[labelformat=empty]{caption}
\usepackage[dvipsnames,table]{xcolor}
\usepackage{multicol,multirow,array,listings,tabularx,lastpage,textcomp,booktabs}

% NOTE(joe): This environment is credit @pnpo (https://tex.stackexchange.com/a/218450)
\lstnewenvironment{algorithm}[1][] %defines the algorithm listing environment
{   
    \lstset{ %this is the stype
        mathescape=true,
        frame=tB,
        numbers=left, 
        numberstyle=\tiny,
        basicstyle=\rmfamily\scriptsize, 
        keywordstyle=\color{black}\bfseries,
        keywords={,procedure, div, for, to, input, output, return, datatype, function, in, if, else, foreach, while, begin, end, }
        numbers=left,
        xleftmargin=.04\textwidth,
        #1
    }
}
{}
\lstnewenvironment{java}[1][]
{   
    \lstset{
        language=java,
        mathescape=true,
        frame=tB,
        numbers=left, 
        numberstyle=\tiny,
        basicstyle=\ttfamily\scriptsize, 
        keywordstyle=\color{black}\bfseries,
        keywords={, int, double, for, return, if, else, while, }
        numbers=left,
        xleftmargin=.04\textwidth,
        #1
    }
}
{}

\newcommand\abs[1]{\lvert~#1~\rvert}
\newcommand{\st}{\mid}

\newcommand{\A}[0]{\texttt{A}}
\newcommand{\C}[0]{\texttt{C}}
\newcommand{\G}[0]{\texttt{G}}
\newcommand{\U}[0]{\texttt{U}}

\newcommand{\cmark}{\ding{51}}
\newcommand{\xmark}{\ding{55}}


\begin{document}
\begin{flushright}
    \StrBefore{\currfilename}{.}
\end{flushright}For a partial ordering, its {\bf Hasse diagram} is a graph whose nodes (vertices) are the elements of the 
domain of the binary relation and which are located such that nodes connected to nodes
above them by (undirected) edges indicate that the relation holds between the lower node and the higher node. 
Moreover, the diagram omits self-loops and omits edges that are guaranteed by transitivity.

\end{document}
\input{../activity-snippets/hasse-diagram-example.tex}
\vfill

{\it Summary}: binary relations can be useful for organizing elements in a domain. 
Some binary relations have special properties that make them act like some familiar relations.
Equivalence relations (reflexive, symmetric, transitive binary relations) ``act like'' equals.
Partial orders (reflexive, antisymmetric, transitive binary relations) ``act like'' less than or equals to.
\newpage
\section*{Review}
\begin{enumerate}
    \item \hspace{1in}\\ \input{../activity-snippets/quiz-binary-relation-same-size.tex} 
    \item \hspace{1in}\\ \input{../activity-snippets/quiz-binary-relation-divisibility.tex}
    \item \hspace{1in}\\ \input{../activity-snippets/quiz-partial-order-hasse.tex}
\end{enumerate}
\newpage

\section*{Wednesday November 24}
\subsection*{Exploring equivalence relations}
\documentclass[12pt, oneside]{article}

\usepackage[letterpaper, scale=0.89, centering]{geometry}
\usepackage{fancyhdr}
\setlength{\parindent}{0em}
\setlength{\parskip}{1em}

\pagestyle{fancy}
\fancyhf{}
\renewcommand{\headrulewidth}{0pt}
\rfoot{\href{https://creativecommons.org/licenses/by-nc-sa/2.0/}{CC BY-NC-SA 2.0} Version \today~(\thepage)}

\usepackage{amssymb,amsmath,pifont,amsfonts,comment,enumerate,enumitem}
\usepackage{currfile,xstring,hyperref,tabularx,graphicx,wasysym}
\usepackage[labelformat=empty]{caption}
\usepackage[dvipsnames,table]{xcolor}
\usepackage{multicol,multirow,array,listings,tabularx,lastpage,textcomp,booktabs}

% NOTE(joe): This environment is credit @pnpo (https://tex.stackexchange.com/a/218450)
\lstnewenvironment{algorithm}[1][] %defines the algorithm listing environment
{   
    \lstset{ %this is the stype
        mathescape=true,
        frame=tB,
        numbers=left, 
        numberstyle=\tiny,
        basicstyle=\rmfamily\scriptsize, 
        keywordstyle=\color{black}\bfseries,
        keywords={,procedure, div, for, to, input, output, return, datatype, function, in, if, else, foreach, while, begin, end, }
        numbers=left,
        xleftmargin=.04\textwidth,
        #1
    }
}
{}
\lstnewenvironment{java}[1][]
{   
    \lstset{
        language=java,
        mathescape=true,
        frame=tB,
        numbers=left, 
        numberstyle=\tiny,
        basicstyle=\ttfamily\scriptsize, 
        keywordstyle=\color{black}\bfseries,
        keywords={, int, double, for, return, if, else, while, }
        numbers=left,
        xleftmargin=.04\textwidth,
        #1
    }
}
{}

\newcommand\abs[1]{\lvert~#1~\rvert}
\newcommand{\st}{\mid}

\newcommand{\A}[0]{\texttt{A}}
\newcommand{\C}[0]{\texttt{C}}
\newcommand{\G}[0]{\texttt{G}}
\newcommand{\U}[0]{\texttt{U}}

\newcommand{\cmark}{\ding{51}}
\newcommand{\xmark}{\ding{55}}


\begin{document}
\begin{flushright}
    \StrBefore{\currfilename}{.}
\end{flushright}A {\bf partition} of a set $A$ is a set of non-empty, disjoint subsets 
$A_1, A_2, \cdots, A_n$ such that 
\[
    A = \bigcup_{i=1}^{n} A_i = \{ x \mid \exists i (x \in A_i) \}
\]
\end{document}
\documentclass[12pt, oneside]{article}

\usepackage[letterpaper, scale=0.89, centering]{geometry}
\usepackage{fancyhdr}
\setlength{\parindent}{0em}
\setlength{\parskip}{1em}

\pagestyle{fancy}
\fancyhf{}
\renewcommand{\headrulewidth}{0pt}
\rfoot{\href{https://creativecommons.org/licenses/by-nc-sa/2.0/}{CC BY-NC-SA 2.0} Version \today~(\thepage)}

\usepackage{amssymb,amsmath,pifont,amsfonts,comment,enumerate,enumitem}
\usepackage{currfile,xstring,hyperref,tabularx,graphicx,wasysym}
\usepackage[labelformat=empty]{caption}
\usepackage[dvipsnames,table]{xcolor}
\usepackage{multicol,multirow,array,listings,tabularx,lastpage,textcomp,booktabs}

% NOTE(joe): This environment is credit @pnpo (https://tex.stackexchange.com/a/218450)
\lstnewenvironment{algorithm}[1][] %defines the algorithm listing environment
{   
    \lstset{ %this is the stype
        mathescape=true,
        frame=tB,
        numbers=left, 
        numberstyle=\tiny,
        basicstyle=\rmfamily\scriptsize, 
        keywordstyle=\color{black}\bfseries,
        keywords={,procedure, div, for, to, input, output, return, datatype, function, in, if, else, foreach, while, begin, end, }
        numbers=left,
        xleftmargin=.04\textwidth,
        #1
    }
}
{}
\lstnewenvironment{java}[1][]
{   
    \lstset{
        language=java,
        mathescape=true,
        frame=tB,
        numbers=left, 
        numberstyle=\tiny,
        basicstyle=\ttfamily\scriptsize, 
        keywordstyle=\color{black}\bfseries,
        keywords={, int, double, for, return, if, else, while, }
        numbers=left,
        xleftmargin=.04\textwidth,
        #1
    }
}
{}

\newcommand\abs[1]{\lvert~#1~\rvert}
\newcommand{\st}{\mid}

\newcommand{\A}[0]{\texttt{A}}
\newcommand{\C}[0]{\texttt{C}}
\newcommand{\G}[0]{\texttt{G}}
\newcommand{\U}[0]{\texttt{U}}

\newcommand{\cmark}{\ding{51}}
\newcommand{\xmark}{\ding{55}}


\begin{document}
\begin{flushright}
    \StrBefore{\currfilename}{.}
\end{flushright}An {\bf equivalence class} of an element $a \in A$ 
with respect to an equivalence relation $R$ on the set $A$ is the set 
\[
    \{s \in A \mid (a, s) \in R \}
\] 
We write $[a]_R$ for this set, which is the equivalence class of $a$ with respect to $R$.
\end{document}
\input{../activity-snippets/partitions-equivalence-classes.tex}
\input{../activity-snippets/congruence-classes-mod-four.tex}
\input{../activity-snippets/modular-arithmetic-motivation.tex}
\input{../activity-snippets/congruence-mod-n-lemma.tex}
\input{../activity-snippets/modular-arithmetic-cycling-examples.tex}
\input{../activity-snippets/modular-arithmetic.tex}
\newpage
\input{../activity-snippets/diffie-helman.tex}
\newpage
\subsection*{Review}
\begin{enumerate}
    \item \hspace{1in}\\ \input{../activity-snippets/quiz-equivalence-relation-properties-proof.tex}
    \item \hspace{1in}\\ \input{../activity-snippets/quiz-modular-exponentiation.tex}
\end{enumerate}

\newpage
\section*{Friday November 26}

No class, in observance of Thanksgiving holiday.
\newpage
\end{document}